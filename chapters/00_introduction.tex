\begin{introduction}
    \label{ch:introduction}
    \section{Motivace}
    V posledních letech roste popularita hry na hudební nástroje a s rozvojem informačních technologií s tím souvisí i snaha uživatelů učit se sami za pomoci veřejně dostupných aplikací. Pro začátečníky se doporučují nástroje buď drnkací (ukulele, kytara) nebo dechové (harmonika, flétna)~\cite{s_2016_the}~\cite{richardson_2019_top}.

    Ukulele oproti kytaře má tu výhodu, že je menší, má o dvě struny méně, struny tolik neřežou do prstů a je tedy jednodušší pro začátečníka. Oproti nástrojům dechovým má hlavně tu výhodu, že se na něj dá zahrát více populárních písní.

    Aktuálně existuje již několik aplikací, ať už webových nebo mobilních, které umožňují uživateli učit se hrát na ukulele sám doma. Problémy těchto aplikací jsou, že jsou buď placené, neobsahují všechny funkcionality, nebo nejsou uživatelsky přívětivé. Rozborem již existujicích řešení se podrobněji zabývá kapitola \nameref{ch:analysis}.

    K této problematice jsem se dostal, jelikož jsem se sám začal učit hrát na ukulele. Již existujicí aplikace mi nevyhovují a to převážně kvůli chybějícím funkcím, nebo neintuitivnímu ovládání. Z tohoto důvodu mi přišlo ideální vytvořit vlastní aplikaci, která bude obsahovat veškeré funkce, chybějící u již existujících aplikací, bude snadno ovladatelná a bude fungovat na počítači i chytrém telefonu.


    \section{Cíle práce}
    Hlavním cílem práce je navrhout, vytvořit a otestovat aplikaci, která umožní začátečníkovi naučit se hrát na ukulele, čehož lze dosáhnout několika způsoby, kterými se budeme zabývat v kapitole \nameref{ch:analysis}. Dílčím cílem je umožnit již zkušenému hráči vyhledat jednotlivé akordy nebo akordy a text k písni.

    Důležitou součástí aplikace je jednoduchost použití, uživatelská přívětivost a zároveň možnost použití některých složitějších mechanismů, jako je například výuka přechytávání akordů s možností zapnutí metronomu, nebo vyhledávání v několika různých množinách (např. akordy, písně a autoři).

    \section{Členění práce}
    Práce je členěna do kapitol odpovídajících krokům vývoje software podle principů metodologie unifikovaného vývoje~\cite[s.~51‑68]{arlow_2007_uml}.

    Kapitola \nameref{ch:analysis} se zabývá průzkumem již existujících aplikací, zjištění cílové skupiny, dostupných technologií a platforem a možnostmi, jak k danému problému přistoupit.

    \hyperref[ch:design]{Druhá kapitola práce}, se věnuje návrhu aplikace. Návrh aplikace je nedílnou součástí vývoje složitějších systémů a aplikací. Umožní programátorovi, nebo týmu programátorů, mít lepší představu o celém projektu, zjistit nedostatky ještě před implementací, nebo například stanovit místa v aplikaci kde může v budoucnu dojít k rozšíření a aplikaci na to řádně připravit. Tato kapitola obsahuje formální stanovení požadavků na aplikaci a vytvoření náhledů uživatelského rozhraní.

    Kapitola \nameref{ch:implementation} rozebírá prostředí v jakém probíhal vývoj aplikace, konečný výběr technologií, určení logického členění a architektury aplikace a některé aspekty zdrojového kódu samotné aplikace. Správně zvolená architektura aplikace usnadní vývoj teď i případné rozšiřování, a může umožnit přepoužití části aplikace při práci na aplikaci jiné.

    \hyperref[ch:testing]{Čtvrtá kapitola} se věnuje testovaním aplikace. Testování je naprosto nezbytné pro vývoj kvalitní aplikace. Vývojář díky tomu zjistí, jakými nedostatky aplikace trpí ještě před tím, než na ně narazí uživatel. Jelikož tyto nedostatky mohou být kritické, může díky nim docházet k úniku citlivých informací a tím poškodit uživatele. Proto se každý systém a aplikace testuje a to na několika úrovních. V rámci této práce se jedná o testy jednotkové, integrační a uživatelské.
\end{introduction}