\begin{introduction}
    \label{ch:introduction}
    \section{Cíle práce}
    Hlavním cílem práce je vytvořit aplikaci, která umožní uživateli se naučit hrát na ukulele bez předchozí znalosti hry, čehož lze dosáhnout několika způsoby, kterými se budeme zabývat v kapitole \nameref{ch:analysis}. Vedlejším cílem je umožnit již zkušenému hráči vyhledat jednotlivé akordy nebo akordy a text k písni.

    Součástí práce je kompletní vývoj software v souladu s pricipy softwarového inženýrství, což zahrnuje analýzu, návrh, implementaci a testování.

    Důležitou součástí aplikace je jednoduchost použití, uživatelská přívětivost a zároveň umožnit použití některých složitějších mechanismů.
    
    \section{Motivace}
    V posledních letech roste popularita hry na hudební nástroje a s rozvojem informačních technologií s tím souvisí i snaha uživatelů se učit sami za pomocí veřejně dostupných aplikací. Pro začátečníky se doporučují nástroje buď drnkací (ukulele, kytara) nebo dechové (harmonika, flétna)\cite{s_2016_the} \cite{richardson_2019_top}.
    
    Ukulele má tu výhodu oproti kytaře, že je menší, má o dvě struny méně, struny tolik neřežou do prstů a je tedy jednodušší pro začátečníka. Oproti nástrojům dechovým má hlavně tu výhodu, že se na něj dá zahrát více populárních písní.

    Aktuálně existuje již několik aplikací, ať už webových nebo mobilních, které umožňují uživateli učit se hrát na ukulele sám doma. Jediný problém těchto aplikací je ten, že jsou buď placené, neobsahují všechny funkcionality, nebo nejsou uživatelsky přívětivé. Rozborem již existujicích řešení se budeme podrobněji zabývat v kapitole \nameref{ch:analysis}.

    K této problematice jsem se dostal, jelikož jsem se sám začal učit hru na ukulele. Již existujicí aplikace mi nevyhovují a to převážně kvůli chybějícím funkcím, nebo neintuitivním ovládáním. Z tohoto důvodu mi přišlo ideální vytvořit vlastní aplikaci, která bude obsahovat veškeré funkce, které mi schází u již existujících aplikací, bude snadno ovladatelná a bude fungovat na počítači i chytrém mobilním telefonu.

    \section{Členění práce}
    Práce je členěna do kapitol odpovídajících krokům vývoje software podle principů metodologie unifikovaného vývoje \cite[s.~51-68]{arlow_2007_uml}.

    Kapitola \nameref{ch:analysis} se zabývá průzkumem již existujících aplikací, zjištění cílové skupiny, dostupných technologií a platforem a možnostmi jak k danému problému přistoupit.

    Druhá kapitola práce, \nameref{ch:design}, se věnuje návrhu aplikace. Návrh aplikace je nedílnou součástí vývoje složitějších systémů a aplikací. Umožní programátorovi, nebo týmu programátorů, mít lepší představu o celém projektu, zjistit nedostatky ještě před implementací, nebo například stanovit místa v aplikaci kde může v budoucnu dojít k rozšíření a aplikaci na to řádně připravit. V rámci této kapitoly je formální stanovení požadavků na aplikaci, vytvoření náhledů uživatelského rozhraní, konečný výběr technologií a určení logického členění a architektury aplikace.

    Třetí kapitola TODO

    V rámci čtvrté kapitoly, \nameref{ch:testing}, je testovaní aplikace. Testování je naprosto nezbytné pro vývoj kvalitní aplikace. Vývojář díky tomu zjistí jakými nedostatky aplikace trpí ještě před tím, než na ně narazí uživatel. Jelikož tyto nedostatky mohou být kritické a může díky nim docházet k úniku citlivých informací a tím poškodit uživatele. Proto se každý systém a aplikace testuje a to na několika úrovních. V rámci této práce se jedná o testy jednotkové, integrační a uživatelské.
\end{introduction}