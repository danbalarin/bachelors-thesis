\section{Hra na ukulele}
\label{sc:ukulele_playing}
Jak již bylo zmíněno, ukulele je čtyřstrunný hudební nástroj, vzhledem podobný malé kytaře. Ukulele původně pochází z Havajských ostrovů, vzniklo z nástroje zvaného Braguinha a o původu jeho jména se vedou spekulace~\cite{rek_2008_kola}.

Samotné hraní je velmi podobné hře na kytaru, ukazováček, prostředníček, prsteníček a malíček levé ruky přitiskávají struny směrem k hmatníku, čímž určují akord a prsty pravé ruky přejíždějí po strunách někde na pomezí krku a těla ukulele. Způsobů jak držet akordy je několik a většinou záleží na dalším, resp. předchozím akordu, aby si hráč zjednodušil přechod. Stejně tak je i více způsobů jak přejíždět prsty po strunách. Prsty mohou jet dolů, nebo nahoru a to břískem nebo nehtem prstu, či lehce klepnout do těla a tím zároveň utlumit struny. Různým kombinacím těchto pohybů se říká \gls{strumming pattern}.

\subsection{Typy ukulele}
\label{ss:ukulele_types}
Ukulele má několik variant, které se odlišují velikostí nástroje a barvou zvuku.

Nejběžnější a zároveň doporučované pro začátečníky je ukulele sopránové. Sopránové ukulele je zároveň nejmenší a jeho běžné ladění je \textit{g,c,e,a} nebo \textit{a,d,f\textsuperscript{\#},h}.

Další ukulele jsou koncertní a tenorové, které jsou větší, ale ladí se stejně jako ukulele sopránové. Rozdíl je tedy v barvě zvuku a velikosti, tedy pohodlnosti držení.

Největší ukulele je barynotové, které je běžně laděné stejně jako vrchní čtyři struny kytary, tedy \textit{d,g,h,e}, a je vhodné pro hráče kteří již umí hrát na kytaru, právě kvůli stejnému ladění.

Další, méně obvyklé verze ukulele jsou šestistrunné, osmistrunné, desetistrunné a uke-benžo (benžolele), které vzniklo spojením ukulele a benža.
