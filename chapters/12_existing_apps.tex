\section{Existující aplikace}
\label{sc:existing_apps}
Mezi alternativy k~této práci lze řadit aplikace zabývající se výukou hry na ukulele nebo kytaru, zpěvníky a metronomy.

\subsection{Výuka hry na ukulele}
\label{ss:existing_teaching_apps}
Většina materiálů pro výuku hry na ukulele je ve formě knih, nebo kurzů.
Kurzy jsou buď osobní nebo online. Mezi největší a nejznámější online aplikace poskytující různé kurzy patří Udemy, který má i kurz hry na ukulele pro začátečníky~\cite{puchmayr_complete}. Alternativou ke knihám a kurzům jsou ještě články nebo krátká nenavazující videa.

Články lze najít například na webu \href{www.kytary.cz}{\emph{kytary.cz}} nebo \href{www.ukuguides.com}{\emph{ukuguides.com}}, kde to je zpracované formou tipů čemu věnovat pozornost a čemu se vyvarovat. Krátká videa zabývající se tipy pro hraní, nebo samotnými skladbami lze najít převážně na portálu \href{www.youtube.com}{\emph{youtube.com}}. Mezi tvůrce s~nejvíce shlédnutými videi měsíčně patří např. Elise Ecklund nebo John Atkins vystupující pod přezdívkou The Ukulele Teacher. Oba autoři mají videa na veškerá témata hry na ukulele, od jeho výběru až po samotné písně s~akordy a jak je hrát. John Atkins je navíc jedním z~tvůrců aplikace The Ukulele App, která slouží hlavně pro výuku hry, ale neobsahuje žádné písně s~akordy a většina funkcionalit je zpoplatněna.

\subsection{Zpěvníky}
\label{ss:songbooks}
Zpěvníky jsou ve většině případů tištěné knihy, pdf soubory k~tisku nebo v~podobě webové či mobilní aplikace. Knižních zpěvníků ukulele je méně než kytarových a ukulele zpěvníků s~českými písněmi je minimum. Soubory k~tisku se dají sehnat i s~českými písněmi, což částečně kompenzuje jejich absenci v~podobě knižní.

Mezi významné webové aplikace se řadí \href{www.ukulele-tabs.com}{\emph{ukulele-tabs.com}} nebo \href{www.ukutabs.com}{\emph{ukutabs.com}}. Mají možnost řazení a vyhledávání ve velkém repertoáru písní. Mezi jejich úskalí patří však absence responzivity (Ukutabs) a absence metronomu, či případné doporučení strumming patternu.

Nejstahovanější aplikace na mobilní telefony jsou např. Ukulele Tabs \& Chords, nebo Ukulele Chords Pocket pro Android a The Ukulele App pro Android a iOS.

\subsection{Metronomy}
\label{ss:metronomes}
Metronomy existují ve formě fyzického zařízení, nebo aplikace, ať už webové či mobilní.

Základní metronom poskytuje i webový vyhledávač \href{www.google.com}{\emph{google.com}}, který ovšem umožňuje pouze nastavení tempa. Mezi jeden z~nejlepších patří metronom webu \href{www.musicca.com}{\emph{musicca.com}}, který umožňuje nastavení tempa a dob, a metronom na webu \href{www.douglasniedt.com}{\emph{douglasniedt.com}}, který sice umožňuje pouze $ \frac{4}{4} $ takt, ale zase zobrazuje doby v~podobě větších dlaždic, které je možné vnímat i periferním viděním, takže uživatel nemusí vnímat tempo jen pomocí zvuků.

Mobilní aplikace zastupuje Metronome Beats a Tuner \& Metronome pro Android a The Metronome by Soundbrenner pro iOS a Android.
