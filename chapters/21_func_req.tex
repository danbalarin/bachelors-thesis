\section{Funkční požadavky}
\label{sc:functional_req}

\noindent \begin{minipage}{\textwidth}
    \paragraph{FP01 Metronom} \label{FP01}
    \begin{smallindent}{}
        Aplikace zobrazí uživateli metronom. Uživatel má možnost si zvolit rychlost, postupné zrychlování a zapnout či vypnout metronom.
    \end{smallindent}
\end{minipage}

\noindent \begin{minipage}{\textwidth}
    \paragraph{FP02 Akordy} \label{FP02}
    \begin{smallindent}{}
        Aplikace zobrazí uživateli akordy, uživatel má možnost nastavit si naladění vlastních strun, uložit toto nastavení, přepínat mezi přednastavenými laděním (sopránové, baritonové...) a vybrat si nějakou podmnožinu akordů, která bude zobrazena.
    \end{smallindent}
\end{minipage}

\noindent \begin{minipage}{\textwidth}
    \paragraph{FP03 Zobrazení strumming patternu} \label{FP03}
    \begin{smallindent}{}
        Aplikace zobrazí uživateli strumming pattern a metronom. Uživatel může ovládat metronom dle \hyperref[FP01]{FP01} a zobrazení strumming patternu reaguje na nastavení rychlosti metronomu.
    \end{smallindent}
\end{minipage}

\noindent \begin{minipage}{\textwidth}
    \paragraph{FP04 Vyhledávání} \label{FP04}
    \begin{smallindent}{}
        Aplikace umožní uživateli vyhledávát písně, akordy a strumming patterny.
    \end{smallindent}
\end{minipage}

\noindent \begin{minipage}{\textwidth}
    \paragraph{FP05 Přechytávání akordů} \label{FP05}
    \begin{smallindent}{}
        Aplikace umožní uživateli si vytvořit vlastní seznam akordů, s volbou nastavení metronomu dle \hyperref[FP01]{FP01}, akordů dle \hyperref[FP02]{FP02} a případně strumming patternu dle \hyperref[FP03-2]{FP03-2}. Zobrazení strumming paternu má možnost vypnout. 
        Přihlášený uživatel má možnost si dané nastavení uložit pod libovolným jménem.
    \end{smallindent}
\end{minipage}

\noindent \begin{minipage}{\textwidth}
    \paragraph{FP06 Přechytávání akordů dle písně} \label{FP06} 
    \begin{smallindent}{}
        \textbf{<<extends>> \hyperref[FP05]{FP05}} \\
        Aplikace zobrazí uživateli přechytávání akordů dle \hyperref[FP05]{FP05} s přednastavenými parametry podle vybrané písně.
    \end{smallindent}
\end{minipage}

\noindent \begin{minipage}{\textwidth}
    \paragraph{FP05 Písně}
    \noindent \paragraph{FP05-1 Vyhledávání} \label{FP05-1}
    \begin{smallindent}{}
        Aplikace zobrazí uživateli katalog písní, uživatel může filtrovat dle obtížnosti, jména, autora, nebo použitých akordů.
        Přihlášený uživatel může filtrovat ještě podle oblíbených.
    \end{smallindent}
\end{minipage}

\noindent \begin{minipage}{\textwidth}
    \paragraph{FP05-2 Zobrazení} \label{FP05-2}
    \begin{smallindent}{}
        Aplikace zobrazí uživateli text písně, akordy a strumming pattern dle \hyperref[FP03-2]{FP03-2}. Uživatel má možnost prokliku na přechytávání akordů dle \hyperref[FP04-2]{FP04-2} a případně odkaz na přehrání písně na serveru třetí strany (např. youtube, spotify).
    \end{smallindent}
\end{minipage}

\noindent \begin{minipage}{\textwidth}
    \paragraph{FP06 Registrace} \label{FP06}
    \begin{smallindent}{}
        Aplikace vytvoří nový učet uživatele.
    \end{smallindent}
\end{minipage}

\noindent \begin{minipage}{\textwidth}
    \paragraph{FP11 Přihlášení} \label{FP11}
    \begin{smallindent}{}
        Aplikace přihlásí uživatele s údaji od uživatele.
    \end{smallindent}
\end{minipage}

\noindent \begin{minipage}{\textwidth}
    \paragraph{FP12 Odhlášení} \label{FP12}
    \begin{smallindent}{}
        Aplikace odhlásí přihlášeného uživatele.
    \end{smallindent}
\end{minipage}

\noindent \begin{minipage}{\textwidth}
    \paragraph{FP13 Nastavení ladění} \label{FP13}
    \begin{smallindent}{}
        Aplikace nastaví ladění ukulele přihlášeného uživatele podle vstupu od uživatele.
    \end{smallindent}
\end{minipage}

\noindent \begin{minipage}{\textwidth}
    \paragraph{FP14 Oblíbení písně} \label{FP14}
    \begin{smallindent}{}
        Aplikace nastaví píseň jako oblíbenou.
    \end{smallindent}
\end{minipage}

\noindent \begin{minipage}{\textwidth}
    \paragraph{FP21 Vytvoření písně} \label{FP21}
    \begin{smallindent}{}
        Aplikace umožňuje moderátorovi vytvořit nový záznam o písni.
    \end{smallindent}
\end{minipage}

\noindent \begin{minipage}{\textwidth}
    \paragraph{FP22 Úprava písně} \label{FP22}
    \begin{smallindent}{}
        Aplikace umožňuje moderátorovi upravit existující záznam o písni.
    \end{smallindent}
\end{minipage}

\noindent \begin{minipage}{\textwidth}
    \paragraph{FP33 Úprava rolí uživatele} \label{FP33}
    \begin{smallindent}{}
        Aplikace umožňuje administrátorovi změnit role uživatele.
    \end{smallindent}
\end{minipage}
