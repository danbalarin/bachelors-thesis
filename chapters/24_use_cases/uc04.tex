\begin{usecase}{PU04 Výuka písní}{uc04}
    Případ užití umožní uživateli zobrazovat písně s texty, akordy, metronomem a doporučeným strumming patternem.

    \begin{scenario}{HS: Zobrazení stránky písně}{uc04:hs}
        \item Scénář začne po přechodu na stránku písničky.
        \item Aplikace zobrazí komponentu z \hyperref[uc03]{PU03}.
        \begin{itemize}
            \item Nastaví strumming pattern, akordy a tempo dle písničky a znemožní editaci těchto paramterů.
        \end{itemize}
        \item Aplikace zobrazí text písně a ovládání automatického odsouvání.
    \end{scenario}
   
    \begin{scenario}{AS01: Zapnutí automatického odsouvání}{uc04:as01}
        \item Scénař začne když uživatel klikne na tlačítko \uv{Start}.
        \item Aplikace spustí automatické odsouvání s nastaveným parametrem.
        \begin{itemize}
            \item Aplikace reaguje na změny parametru okamžitě.
        \end{itemize}
        \item Tlačitko \uv{Start} se změní na tlačítko \uv{Stop} a scénář končí.
    \end{scenario}

    \begin{scenario}{AS02: Vypnutí automatického odsouvání}{uc04:as02}
        \item Scénář začne když uživatel klikne na tlačítko \uv{Stop}.
        \item Aplikace vypne automatické odsouvání.
        \item Tlačítko \uv{Stop} se změní na \uv{Start}.
    \end{scenario}

    \begin{scenario}{AS03: Oblíbení písně}{uc04:as03}
        \item Scénář začne při zobrazení komponenty pokud je uživatel přihlášen.
        \item Aplikace zobrazí tlačítko \uv{Oblíbit}.
        \begin{itemize}
            \item Pokud uživatel na tlačítko klikne a píseň nemá oblíbenou, pak aplikace nastaví píseň jako oblíbenou.
            \item Pokud uživatel na tlačítko klikne a píseň má oblíbenou, pak aplikace odebere píseň jako oblíbenou.
        \end{itemize}
    \end{scenario}
\end{usecase}
