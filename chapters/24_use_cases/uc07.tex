\begin{usecase}{PU07 Správa uživatelského profilu}{uc07}
    Případ užití umožní uživateli si zobrazit vlastní uživatelský profil a upravovat některé hodnoty.

    \begin{scenario}{HS: Zobrazení profilu}{uc07:hs}
        \item Scénář začne když uživatel zobrazí tuto komponentu.
        \item Aplikace zobrazí uživatelský profil s předvyplněnými polemi.
    \end{scenario}

    \begin{scenario}{AS01: Úprava hodnot}{uc07:as01}
        \item Scénář začne, když uživatel změní hodnotu v některém z polí.
        \item Aplikace umožní upravit hodnoty některých polí.
        \item Aplikace zobrazí tlačítko sloužící k uložení informací \enquote{Save}.
        \item Po kliknutí na tlačítko \enquote{Save} aplikace provede validace polí podle \nameref{uc07:alg1}.
        \begin{itemize}
            \item Pokud validace proběhnou úspěšně, pak jsou informace odeslány na server a pokračuje se \hyperref[uc07:as02]{AS02}.
            \item Jinak jsou zobrazeny chybové hlášky podle toho, jaké validace selhaly.
        \end{itemize}
        \item Pokud chce uživatel opustit stránku před uložením změněných hodnot, tak je o tom informován s možností uložit hodnoty.
    \end{scenario}

    \begin{scenario}{AS02: Přenačtení hodnot}{uc07:as02}
        \item Scénář začne po kliknutí na tlačítko \enquote{Refresh}.
        \item Aplikace přenačte hodnoty ze serveru a aktualizuje pole.
        \item Scénář končí.
    \end{scenario}

    \begin{scenario}{Alg.1}{uc07:alg1}
        \item Pokud hodnota v poli \enquote{Email} není platný email, pak algoritmus vrátí chybu.
        \item Pokud je hodnota v poli \enquote{Heslo} kratší než 5 znaků, pak algoritmus vrátí chybu.
        \item Pokud hodnota v poli \enquote{Opakování hesla} neodpovídá hodnotě v poli \enquote{heslo}, pak algoritmus vrátí chybu.
    \end{scenario}
\end{usecase}