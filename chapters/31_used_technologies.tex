\section{Použité technologie}
\label{sc:used_techologies}
Vývoj webových aplikací v dnešní době je v rozkvětu, komunita se stále zvětšuje a velké technologické firmy přesouvají desktopové aplikace na web (např. Skype, nebo Outlook).

Pro vývoj samotné aplikace využijeme Typescript, pro frontend framework React v kombinaci s Apollo Client a pro backend Node.JS, a konkrétně knihovnu Express a Apollo Server a pro komunikaci GraphQL. K vývoji však patří ještě další podpůrné nástroje, které budeme probírat v nasledujících sekcích, stejně jako podrobnější popis již zmíněné technologie.

\subsection{GraphQL}
\label{ss:graphql}
GraphQL, kde \emph{QL} je zkratka pro Query Language, je dotazovací jazyk. Jedná se o jazyk vyvíjený přímo pro použití v \acrshort{api}, původně vyvíjený firmou \emph{Facebook}, dnes již pod neziskovou organizací \emph{GraphQL Foundation}. Na rozdíl od standardního \acrshort{rest} API, které umožňuje získání, nebo úpravu jednoho objektu pomocí specifické cesty, tak GraphQL server naslouchá jedné cestě a jako parametr dostává stromovou strukturu zapsanou ve speciálním formátu. Tento přístup má mnoho výhod, umožňuje uživateli vybrat si přímo jaká data požaduje (a tím snížit datový tok), slučování více požadavků do jednoho a silné typování samotného API endpointu. Nevýhoda oproti REST API, je využití jedné cesty a tím nemožnost cachování na straně prohlížeče a celková komplexita systému.

\subsection{Yarn v2}
\label{ss:yarn}
Yarn je alternativa k \acrshort{npm}, což je balíčkovací systém pro Javascript, obdoba NuGetu pro C\# nebo Mavenu pro Javu. Od verze 2.0, taky zvané \emph{berry}, integruje podporu tzv. monorepa, tedy přístupu kdy jeden repozitář obsahuje několik podprojektů. Tuto možnost dříve poskytoval nástroj \href{https://github.com/lerna/lerna}{\emph{lerna}}. To vede k možnosti modularizace aplikace a zavedení Single responsibility principu.

Další podstatná funkcionalita Yarn v2 je vynucená podpora \acrfull{pnp} režimu. \acrshort{pnp} je přístup zavedený týmem hlavních vývojářů Yarnu, jde o to, že při standartní instalaci balíčků pomocí npm nebo yarn bez pnp, jsou balíčky staženy a rozbaleny do adresáře \emph{node\_modules} ze kterého potom čerpá \acrfull{nra} \cite{joyentinc_1_noderesolutionalgorithm}. Tento algoritmus při sestavování výsledné aplikace pro každé volání funkce \emph{require()} vždy projde složku \emph{node\_modules} a pokud daný balíček nenajde, pak rekurzivně pokračuje v nadřazeném adresáři (za předpokladu, že nadřazený adresář obsahuje složku \emph{node\_modules}). Tento přístup je i s nějakými optimalizacemi velmi pomalý. Yarn s \acrshort{pnp} provádí to, že při prvotní instalaci závislostí místo vytvoření \emph{node\_modules} složky, kam by se následně rozbalovaly balíčky, tak vytvoří vlastní adresář \emph{.yarn/cache} kam stáhne balíčky, nerozbaluje je a následně vytvoří soubor \emph{pnp.js} který uchovává odkazy na všechny balíčky a závislosti. Výhodou je až o 70\% vyšší rychlost a menší zatížení disku a procesoru při instalaci. Nevýhodou je, že všechny instalované balíčky musí mít explicitní výpis všech svých závislostí, což velká část do dnešní doby nemá a spoléhá na to, že jejich závislosti prostě budou k dispozici a \acrshort{nra} je najde.

Jelikož Yarn v2 ještě nevyšel jako stabilní verze, nýbrž pouze jeho release kandidáti, tak se k němu špatně hledají informace v případě potíží. Další problém který tento přístup přinesl do práce byla nutnost opravy chybějících závislostí některých balíčků. Jak již bylo zmíněno, každý balíček musí mít seznam svých závislostí a když nemá, tak sestavení selže. Takže je na každém uživateli aby tyto závislosti doplnil a to formou zápisu do konfiguračního \emph{.yarnrc.yml} souboru, kde je v práci takto opraveno zhruba 30 závislostí. Výhodou pak je mnohem rychlejší instalace závislostí, sestavení aplikace a podpora monorepa.

\subsection{Express}
\label{ss:express}
Express je minimalistický webový framework pro Node.js. Jedná se o jednoduchý webový server, který dokáže obsluhovat standardní http požadavky. Umožňuje, mimo jiné, i jednotlivým cestám nastavit middleware, což jsou funkce které jsou spuštěny před samotným vyhodnocováním funkce k dané cestě. Typickým využitím je autentizace uživatele, nebo zapisování do log souboru.

\subsection{Apollo Server}
\label{ss:apollo_server}
Apollo server je GraphQL server vyvíjený společností \emph{Meteor Development Group, Inc.}, který je nezávislý na volbě webového serveru, v našem případě \ref{ss:express}, který je takzvaně \emph{unopinionated}, což znamená, že vývojáře netlačí nějakou specifickou cestou vývoje.

\subsection{React}
\label{ss:react}

\subsection{Apollo client}
\label{ss:apollo_client}

\subsection{Storybook.js}
\label{ss:storybook}

\subsection{Chakra UI}
\label{ss:chakra_ui}

\subsection{Styled components}
\label{ss:express}

\subsection{Jest}
\label{ss:jest}
