\section{Použité technologie}
\label{sc:used_techologies}
Pro vývoj samotné aplikace byl využit TypeScript, pro frontend framework React v~kombinaci s~Apollo Client\footnote{https://www.apollographql.com/} a pro backend Node.JS, konkrétně knihovnu Express\footnote{https://expressjs.com/} a Apollo Server\footnote{https://www.apollographql.com/} a pro komunikaci GraphQL. K~vývoji však patří ještě další podpůrné nástroje, kterými se zabývají následující sekce, stejně jako podrobnějšímu popisu již zmíněných technologií.

\subsection{GraphQL}
\label{ss:graphql}
GraphQL, kde \emph{QL} je zkratka pro Query Language, je dotazovací jazyk. Jedná se o~jazyk vyvíjený přímo pro použití v~\acrshort{api}, původně vyvíjený firmou \emph{Facebook}, dnes již pod neziskovou organizací \emph{GraphQL Foundation}. Na rozdíl od standardního \acrshort{rest} API, které umožňuje získání, nebo úpravu jednoho objektu pomocí specifické cesty, tak GraphQL server naslouchá jedné cestě a jako parametr dostává stromovou strukturu zapsanou ve speciálním formátu. Tento přístup má mnoho výhod, umožňuje vývojáři vybrat si přímo jaká data požaduje (a tím snížit datový tok), také umožňuje slučování více požadavků do jednoho, silné typování samotného \acrshort{api} endpointu což umožňuje našeptávání a statickou kontrolu. Nevýhoda oproti REST \acrshort{api} je využití jedné cesty a tím znemožnění cachování na straně prohlížeče. \cite{brito2020rest}

Ve spojitosti s~GraphQL se budeme bavit o~požadavcích typu \uv{query} a \uv{mutace}. Query je obdobou GET požadavku REST \acrshort{api}, tedy pouze získá data. Mutace na druhou stranu je obdoba POST, UPDATE nebo DELETE požadavku, tedy jakákoliv operace, která nějak manipuluje s~daty. Oba typy mohou mít proměnné a vracet nějaké hodnoty. Existuje ještě jeden specifický typ požadavku, \uv{subscription}, který umožňuje navázat trvalé spojení a dostávat okamžitá upozornění na nová data. Tento typ požadavku avšak není potřebný pro tento typ aplikace, hodí se však kdekoliv, kde je třeba upozornit na změny okamžitě, např. chat. \cite{porcello_2018_learning}

\subsection{MongoDB a Mongoose}
\label{ss:mongoose}
MongoDB\footnote{https://www.mongodb.com/} je NoSQL databáze. Hlavní výhoda NoSQL databází je rychlejší vývoj a prototypování, jelikož se nemusí žádným způsobem vytvářet tabulky. Výhodou MongoDB je přístup k~ukládání dat, a to ve formě dokumentů, kde každý dokument je tvořen z~dvojic klíče a hodnoty. Takovýto dokument je až na typy shodný s~JSON, tedy propojení MongoDB a javascriptové aplikace je triviální.

Mongoose\footnote{https://mongoosejs.com/} je tzv. \acrfull{orm} nástroj vytvořený firmou \emph{Automattic} přímo pro databázi MongoDB. To umožňuje vytvořit jednotlivé datové modely ze schémat. Schéma je objekt definující vlastnosti modelu, jako jsou pole, indexy, validace, virtuální pole a další \cite{automatticinc_2020_mongoose}. Takto vytvořený model poskytuje možnost vytvářet, vyhledávat, upravovat či mazat záznamy v~databázi.

Další a největší výhodou tohoto přístupu vytváření modelů je možnost z~takových modelů automaticky vytvořit GraphQL mutace a query a to za použití knihoven \emph{graphql-compose} a \emph{graphql-compose-mongoose}. Tyto knihovny vytvoří základní query a mutace pro vyhledávání či úpravu modelů, které jsou snadno rozšiřitelné o~další, vlastní, operace nad modely.

\subsection{Yarn v2}
\label{ss:yarn}
Yarn\footnote{https://yarnpkg.com/} je alternativa k~\acrshort{npm}, což je balíčkovací systém pro JavaScript, obdoba NuGetu pro C\# nebo Mavenu pro Javu.

Od verze 2.0, taky zvané \emph{berry}, integruje podporu tzv. monorepozitáře, tedy přístupu kdy jeden repozitář obsahuje několik podprojektů. Tuto možnost poskytuje i nástroj \href{https://github.com/lerna/lerna}{\emph{Lerna}}, kde se vývojáři~\nameref{ss:yarn} inspirovali. To vede k~možnosti modularizace aplikace a zavedení Single responsibility principu.

Další podstatná funkcionalita Yarn v2 je vynucená podpora \acrfull{pnp} režimu. \acrshort{pnp} je přístup zavedený týmem hlavních vývojářů Yarnu. Při standartní instalaci balíčků pomocí npm nebo yarn bez pnp, jsou balíčky staženy a rozbaleny do adresáře \emph{node\_modules} ze kterého potom čerpá \acrfull{nra}~\cite{joyentinc_1_noderesolutionalgorithm}. Tento algoritmus při sestavování výsledné aplikace pro každé volání funkce \emph{require()} vždy projde složku \emph{node\_modules} a pokud daný balíček nenajde, pak rekurzivně pokračuje v~nadřazeném adresáři (za předpokladu, že nadřazený adresář obsahuje složku \emph{node\_modules}). Tento přístup je i s~nějakými optimalizacemi velmi pomalý. Yarn s~\acrshort{pnp} provádí to, že při prvotní instalaci závislostí místo vytvoření \emph{node\_modules} složky, kam by se následně rozbalovaly balíčky, tak vytvoří vlastní adresář \emph{.yarn/cache} kam stáhne balíčky, nerozbaluje je a následně vytvoří soubor \emph{pnp.js} který uchovává odkazy na všechny balíčky a závislosti. Výhodou je až o~70\% vyšší rychlost a menší zatížení disku a procesoru při instalaci. Nevýhodou je, že všechny instalované balíčky musí mít explicitní výpis všech svých závislostí, což velká část do dnešní doby nemá a spoléhá na to, že jejich závislosti prostě budou k~dispozici a \acrshort{nra} je najde.

Jelikož Yarn v2 ještě nevyšel jako stabilní verze, nýbrž pouze jeho release kandidáti, tak se k~němu špatně hledají informace v~případě potíží. Další problém, který tento přístup přinesl do práce, byla nutnost opravy chybějících závislostí některých balíčků. Jak již bylo zmíněno, každý balíček musí mít seznam svých závislostí a když nemá, tak sestavení selže. Takže je na každém vývojáři, aby tyto závislosti doplnil a to formou zápisu do konfiguračního \emph{.yarnrc.yml} souboru, kde je v~práci takto opraveno zhruba 30 závislostí. Výhodou pak je mnohem rychlejší instalace závislostí, sestavení aplikace a podpora monorepozitáře.

\subsection{Express}
\label{ss:express}
Express je minimalistický webový framework pro Node.js. Jedná se o~jednoduchý webový server, který dokáže obsluhovat standardní http požadavky. Umožňuje, mimo jiné, jednotlivým cestám nastavit middleware, což jsou funkce, které jsou spuštěny před samotným vyhodnocováním funkce k~dané cestě. Typickým využitím je autentizace uživatele, nebo zapisování do log souboru. V~rámci aplikace byl použit jak pro poskytování backend, tak i frontend serveru. Backend server poskytuje GraphQL \acrshort{api} a zajišťuje napojení na databázi, zatím co frontend server poskytuje \acrfull{ssr} aplikace. Co znamená \acrshort{ssr} se budeme podrobněji zabývat v~rámci samostatné sekce~\ref{ss:ssr}.

\subsection{Apollo Server}
\label{ss:apollo_server}
Apollo Server je GraphQL server vyvíjený společností \emph{Meteor Development Group, Inc.}, který je nezávislý na volbě webového serveru, v~našem případě~\ref{ss:express}, který je takzvaně \emph{unopinionated}, což znamená, že vývojáře netlačí nějakou specifickou cestou vývoje. Jeho hlavní výhodou je hlavně jednoduchost, ale i popularita, tedy větší komunita. Menším detailem, který potěší hlavně vývojáře, je automatické vystavování GraphQL Playground, což je webová aplikace umožňující simulování dotazů, které má funkce jako IDE, tedy našeptávání, kontrola typů, kontrola syntaxe, \ldots{}.

\subsection{Apollo Client}
\label{ss:apollo_client}
Apollo Client\footnote{https://www.apollographql.com/} je od stejné společnosti jako~\ref{ss:apollo_server} a tyto dvě knihovny nejlépe fungují právě, když jsou pohromadě. Mezi hlavní výhody Apollo Client patří snadné cachování, velké množství dalších balíčků od komunity, nebo správa lokálního stavu aplikace, kterému se budeme věnovat podrobněji v~sekci~\ref{ss:local_state_management}.

\subsection{React}
\label{ss:react}
Konečná volba padla na React hlavně z~důvodu největší popularity, a tedy i největší podpory ze strany komunity. V~rámci této práce je již zakomponováno několik technologíí, které nejsou zatím se stabilním stavu a je tedy mít nějaký styčný bod, na který je možný plně spoléhat. I~z~toho důvodu byla použita stabilní verze Reactu a ne \emph{nightly} nebo \emph{experimentální}, které sice poskytují některé pěkné funkcionality, např. \emph{Concurrent Mode} zrychlující render aplikace nebo \emph{Suspense} zjednodušující čekání na asynchronní úkoly, ale výsledná aplikace by byla nestabilní, což by pro vývoj znamenalo značné zdržení.


\subsection{Storybook.js}
\label{ss:storybook}
Tato knihovna umožňuje vytvářet interaktivní dema k~jednotlivým React komponentám. Každá komponenta může obsahovat story soubor, který specifikuje, jak se má daná komponenta zobrazit a případně nastavení k~dalším pluginům. Tyto story soubory jsou následně načteny, seřazeny do stromové struktury a z~toho je následně vytvořena interaktivní dokumentace. Tato dokumentace je velmi přehledná a zároveň nutí vývojáře izolovat komponenty, což opět vede implementaci Single responsibility principu. Jednotlivé pluginy rozšiřují možnosti například pro generování dokumentace z~JSDoc komentářů nebo aby jednotlivé story byly součástí jednotkových testů.

\subsection{Jest}
\label{ss:jest}
Jest\footnote{https://jestjs.io/} je testovací knihovna pro JavaScript vyvíjená společností Facebook. Umožnuje psát jednotkové testy, simulovat data, generovat reporty pokrytí kódu ale hlavně, má ze všech testovacích knihoven pro JavaScript nejlepší podporu testování React komponent.
