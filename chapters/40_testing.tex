\chapter{Testování}
\label{ch:testing}


\section{Budoucí vývoj}
\label{sc:upcomming_development}
Aplikace je nyní ve fázi funkčního prototypu, jsou implementovány všechny základní požadavky, ale stále je zde prostor na zlepšení. V následujících sekcích jsou rozebrány některé takové faktory.

\subsubsection{Data}
To co nejvíce aplikaci schází v tuto dobu jsou funkční data. Tím jsou myšleny písně s texty, akordy atd. Tyto data se dají jednoduše najít, ale pro osobní použití. Bohužel není žádné API poskytující seznam písní s textem a akordy pro ukulele. Tato data by se dala získat buď přepisem z jiných stránek či jejich scrapovaním, což by už mělo přesah do autorského zákona. Další možnost je kontaktovat přímo vydavatele a uzavřít spolupráci, která avšak nejspíš bude velmi finančně náročná. Poslední možnost by byla uzavřít spolupráci s již existujícímí aplikacemi a získat data od nich.

\subsubsection{Úpravy aplikace}
Hlavním neduhem stávající aplikace je hlavně design. Aplikace je sice ve dvou barevných schématech, ty však nejsou

\subsubsection{Nové funkce}
\subsubsection*{Překlad}

\subsubsection*{Podpora vybrnkávání}

\subsubsection*{Nastavení ladění}
Tato funkcionalita je témeř imlementována. Komponenta vykreslovaného akordu dostává na vstupu mimo akordu k vykreslení i aktuální ladění, které je vždy nastaveno na \emph{gcea}. Rozšíření by tedy spočívalo v přidání možnosti uživateli si toto ladění změnit, případně uložit pro příští návštěvy.

