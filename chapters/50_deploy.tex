\chapter{Kontinuální integrace a nasazení}
\label{ch:ci_cd}
Kontinuální integrace, dodání a nasazení jsou přístupy k~automatizaci jednotlivých procesů, které jsou mezi vývojem a nasazením aplikace. V~rámci této práce si pouze nastíníme, co jednotlivé výrazy znamenají.

Kontinuální integrace se zabývá automatickou validací kódu pomocí automatizovaných testů. Tyto testy jsou spuštěny ve chvíli kdy programátor nahraje do systému správy verzí (\acrshort{scm}). Tyto automatické testy ověří funkčnost, správnost a další požadavky na aplikaci a následně zpětně informují vývojáře, resp. tým o~výsledcích testů. V~praxi se to běžně využívá před zapojením nové části kódu do celé aplikace. Výhodou tohoto přístupu je minimalizování chyb a zvýšení produktivity programátora. V~této práci kontinuální integrace byla realizována pomocí \emph{GitHub Actions}. Jelikož je celá aplikace open source a uložena na serveru \href{www.github.com}{github.com}, tak má nárok na využití systému GitHub Actions, který právě umožňuje automatické spouštění testů. Konkrétní testy prováděné automaticky jsou~\hyperref[sc:unit_tests]{jednotkové} a následně smoke build storybooku, který pouze ověří, zdali lze sestavit, tedy zdali aplikace lze sestavit.~\cite[s.~7]{rossel_2017_continuous}

Kontinuální dodání a nasazení jsou pojmy velmi podobné. Kontinuální dodání vytvoří balíčky aplikace připravené k~nasazení a kontinuální nasazení je rovnou i nasadí. Jak dodání, tak nasazení je spouštěno automaticky po úspěšném provedení kontinuální integrace, nebo jiných krocích. V~praxi to může být manuální spuštění na konci sprintu nebo automaticky po zařazení kódu do hlavní větve.~\cite[s.~18]{rossel_2017_continuous} Při tvorbě této aplikace bylo využito obojího. Při každé změně k~hlavní větvi jsou spuštěny dva nasazovací skripty. První zajišťuje automatické sestavení dokumentace a statické verze storybooku a nasazení na \emph{GitHub Pages}. Tato dokumentace je veřejně dostupná a hostovaná na serveru GitHub\footnote{\href{https://danbalarin.github.io/ukulele-learning-site/}{danbalarin.github.io/ukulele-learning-site/}}. Druhý sestavuje výslednou aplikaci, vytváří docker kontejnery a následně spustí aktualizaci na vzdáleném serveru. Docker kontejnery jsou vytvořeny dva, jeden pro backend a druhý pro frontend a jsou uloženy na veřejném docker repozitáři\footnote{\href{https://hub.docker.com/repository/docker/kenny11/uls-react}{hub.docker.com/repository/docker/kenny11/uls-react}}\footnote{\href{https://hub.docker.com/repository/docker/kenny11/uls-nodejs}{hub.docker.com/repository/docker/kenny11/uls-nodejs}}. Následně se skript připojí přes ssh k~vzdálenému serveru, spustí jednoduchý skript, který zastaví běžící instance, smaže, nahraje nové a spustí je. Jako poslední krok je spuštěn jednoduchý smoke test~\ref{code:smoke}, který ověří, zdali frontend odpovídá na portu 80, a backend odpovídá na portu 4000.

\begin{figure}[h!]
    \centering
    \begin{minted}{bash}
curl -sSL --max-time 5 -D - $URL:$PORT -o /dev/null
    \end{minted}
    \caption{Smoke test ověřující funkčnost http serveru}
    \label{code:smoke}
\end{figure}
