\begin{conclusion}
    Cílem práce bylo vytvořit aplikaci pro podporu výuky hry na ukulele v~souladu s~metodami softwarového inženýrství. Ve zkratce jsem uvedl, co je potřeba znát pro hraní na ukulele. Prozkoumal jsem již existující aplikace a zvážil jejich výhody a nevýhody. Dále jsem uvedl dostupné technologie a cílovou skupinu, pro kterou je aplikace tvořena. Na základě těchto poznatků jsem stanovil funkční a nefunkční požadavky, ze kterých jsem stanovil případy užití, wireframy a diagram tříd. Následně jsem zvolil technologie, které jsem využil k~vytvoření aplikace. Popsal jsem metodiky použité k~návrhu architektury aplikace a určil logické členění částí aplikace. Na základě návrhu jsem implementoval prototyp aplikace a automatické testy aplikace. Poté jsem vytvořil testovací a nasazovací skripty, které pracují zcela automaticky.

    Prototyp aplikace je plně funkční, umožňuje vyhledávat akordy, strumming patterny, písně i autory. Uživatel má možnost se přihlásit a tím zpřístupnit nastavování písní jako oblíbené. Moderátor může přidávat, či upravovat písně a autory a administrátor může měnit role vytvořených účtů. Aplikace je dostupná v~anglickém jazyce.

    Aplikaci lze rozšířit o~další funkcionality, třeba vybrnkávání, či nastavování vlastního ladění ukulele. Vhodné by bylo aplikaci rozšířit o~další jazyky či vylepšit grafické pojetí. Dobrým nápadem je z~responzivní aplikace udělat PWA.

    Hlavní přínos mé práce spočívá hlavně v~přehlednosti a jednoduchosti aplikace, z~čehož benefitují hlavně uživatelé méně zkušení s~počítačem. Zvolené technologie umožnily rychle a efektivně vytvořit aplikaci, kterou je možné jednoduše rozšířit, či upravit. Jediná nevýhoda zvolených technologií je špatná kompatibilita, což znemožnilo integrační testování. Tento problém by však měl v~budoucnu s~rostoucí komunitou vymizet a aplikaci tedy bude možné o~dané testy rozšířit a tím zajistit konzistenci kvality.
\end{conclusion}