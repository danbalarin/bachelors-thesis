% arara: xelatex
% arara: xelatex
% arara: xelatex

\documentclass[thesis=B,czech]{template/FITthesis}[2019/03/06]
\selectlanguage{czech}

\usepackage[utf8]{inputenc}

\usepackage{dirtree}

% % list of acronyms
\usepackage[acronym,nonumberlist,toc,numberedsection=autolabel]{glossaries}
\iflanguage{czech}{\renewcommand*{\acronymname}{Seznam pou{\v z}it{\' y}ch zkratek}}{}
\makeglossaries

\newcommand{\tg}{\mathop{\mathrm{tg}}} %cesky tangens
\newcommand{\cotg}{\mathop{\mathrm{cotg}}} %cesky cotangens

\department{Katedra softwarového inženýrství}
\title{Specializovaná webová aplikace pro výuku hry na ukulele}
\authorGN{Dan} %(křestní) jméno (jména) autora
\authorFN{Balarin} %příjmení autora
\authorWithDegrees{Dan Balarin} %jméno autora včetně současných akademických titulů
\author{Dan Balarin} %jméno autora bez akademických titulů
\supervisor{Ing. Marek Suchánek}
\acknowledgements{TODO\: Doplňte, máte-li komu a za co děkovat. V~opačném případě úplně odstraňte tento příkaz.}
\abstractCS{TODO\: V~několika větách shrňte obsah a přínos této práce v~češtině. Po přečtení abstraktu by se čtenář měl mít čtenář dost informací pro rozhodnutí, zda chce Vaši práci číst.}
\abstractEN{TODO\:Sem doplňte ekvivalent abstraktu Vaší práce v~angličtině.}
\placeForDeclarationOfAuthenticity{V~Praze}
\declarationOfAuthenticityOption{4} %volba Prohlášení (číslo 1-6)
\keywordsCS{TODO\: Nahraďte seznamem klíčových slov v češtině oddělených čárkou.}
\keywordsEN{TODO\: Nahraďte seznamem klíčových slov v angličtině oddělených čárkou.}
\website{https://github.com/danbalarin/bachelors-thesis} %volitelná URL práce, objeví se v tiráži - úplně odstraňte, nemáte-li URL práce

\begin{document}

\begin{introduction}
    \label{ch:introduction}
    \section{Motivace}
    V posledních letech roste popularita hry na hudební nástroje a s rozvojem informačních technologií s tím souvisí i snaha uživatelů učit se sami za pomoci veřejně dostupných aplikací. Pro začátečníky se doporučují nástroje buď drnkací (ukulele, kytara) nebo dechové (harmonika, flétna)~\cite{s_2016_the}~\cite{richardson_2019_top}.

    Ukulele oproti kytaře má tu výhodu, že je menší, má o dvě struny méně, struny tolik neřežou do prstů a je tedy jednodušší pro začátečníka. Oproti nástrojům dechovým má hlavně tu výhodu, že se na něj dá zahrát více populárních písní.

    Aktuálně existuje již několik aplikací, ať už webových nebo mobilních, které umožňují uživateli učit se hrát na ukulele sám doma. Problémy těchto aplikací jsou, že jsou buď placené, neobsahují všechny funkcionality, nebo nejsou uživatelsky přívětivé. Rozborem již existujicích řešení se podrobněji zabývá kapitola \nameref{ch:analysis}.

    K této problematice jsem se dostal, jelikož jsem se sám začal učit hrát na ukulele. Již existujicí aplikace mi nevyhovují a to převážně kvůli chybějícím funkcím, nebo neintuitivnímu ovládání. Z tohoto důvodu mi přišlo ideální vytvořit vlastní aplikaci, která bude obsahovat veškeré funkce, chybějící u již existujících aplikací, bude snadno ovladatelná a bude fungovat na počítači i chytrém telefonu.


    \section{Cíle práce}
    Hlavním cílem práce je navrhout, vytvořit a otestovat aplikaci, která umožní začátečníkovi naučit se hrát na ukulele, čehož lze dosáhnout několika způsoby, kterými se budeme zabývat v kapitole \nameref{ch:analysis}. Dílčím cílem je umožnit již zkušenému hráči vyhledat jednotlivé akordy nebo akordy a text k písni.

    Důležitou součástí aplikace je jednoduchost použití, uživatelská přívětivost a zároveň možnost použití některých složitějších mechanismů, jako je například výuka přechytávání akordů s možností zapnutí metronomu, nebo vyhledávání v několika různých množinách (např. akordy, písně a autoři).

    \section{Členění práce}
    Práce je členěna do kapitol odpovídajících krokům vývoje software podle principů metodologie unifikovaného vývoje~\cite[s.~51‑68]{arlow_2007_uml}.

    Kapitola \nameref{ch:analysis} se zabývá průzkumem již existujících aplikací, zjištění cílové skupiny, dostupných technologií a platforem a možnostmi, jak k danému problému přistoupit.

    \hyperref[ch:design]{Druhá kapitola práce}, se věnuje návrhu aplikace. Návrh aplikace je nedílnou součástí vývoje složitějších systémů a aplikací. Umožní programátorovi, nebo týmu programátorů, mít lepší představu o celém projektu, zjistit nedostatky ještě před implementací, nebo například stanovit místa v aplikaci kde může v budoucnu dojít k rozšíření a aplikaci na to řádně připravit. Tato kapitola obsahuje formální stanovení požadavků na aplikaci a vytvoření náhledů uživatelského rozhraní.

    Kapitola \nameref{ch:implementation} rozebírá prostředí v jakém probíhal vývoj aplikace, konečný výběr technologií, určení logického členění a architektury aplikace a některé aspekty zdrojového kódu samotné aplikace. Správně zvolená architektura aplikace usnadní vývoj teď i případné rozšiřování, a může umožnit přepoužití části aplikace při práci na aplikaci jiné.

    \hyperref[ch:testing]{Čtvrtá kapitola} se věnuje testovaním aplikace. Testování je naprosto nezbytné pro vývoj kvalitní aplikace. Vývojář díky tomu zjistí, jakými nedostatky aplikace trpí ještě před tím, než na ně narazí uživatel. Jelikož tyto nedostatky mohou být kritické, může díky nim docházet k úniku citlivých informací a tím poškodit uživatele. Proto se každý systém a aplikace testuje a to na několika úrovních.

    \hyperref[ch:ci_cd]{Poslední kapitola} probírá automatické testování a nasazování. Automatizace testů a nasazování zrychluje vývoj a celkovou workflow týmu. Tato automatizace deleguje spouštění testů a buildů aplikace na dedikovaný server, takže se vývojář může zatím věnovat něčemu jinému.
\end{introduction}

\chapter{Analýza}
\label{ch:analysis}
Proces vývoje aplikace začíná u~analýzy. V~rámci analýzy se zkoumá daná problematika, kterou má aplikace řešit, již existující řešení, cílová skupina a možnosti řešení. Tato část je důležitá hlavně z~toho důvodu, abychom náhodou neřešili problém, který už někdo vyřešil za nás, nebo nevytvářeli aplikaci pro jinou cílovou skupinu, než bychom měli.

\section{Hra na ukulele}
\label{sc:ukulele_playing}
Jak již bylo zmíněno, ukulele je čtyřstrunný hudební nástroj, vzhledem podobný malé kytaře. Ukulele původně pochází z Havajských ostrovů, vzniklo z nástroje zvaného Braguinha a o původu jeho jména se vedou spekulace~\cite{rek_2008_kola}.

Samotné hraní je velmi podobné hře na kytaru, ukazováček, prostředníček, prsteníček a malíček levé ruky přitiskávají struny směrem k hmatníku, čímž určují akord a prsty pravé ruky přejíždějí po strunách někde na pomezí krku a těla ukulele. Způsobů jak držet akordy je několik a většinou záleží na dalším, resp. předchozím akordu, aby si hráč zjednodušil přechod. Stejně tak je i více způsobů jak přejíždět prsty po strunách. Prsty mohou jet dolů, nebo nahoru a to břískem nebo nehtem prstu, či lehce klepnout do těla a tím zároveň utlumit struny. Různým kombinacím těchto pohybů se říká \gls{strumming pattern}.

\subsection{Typy ukulele}
\label{ss:ukulele_types}
Ukulele má několik variant, které se odlišují velikostí nástroje a barvou zvuku.

Nejběžnější a zároveň doporučované pro začátečníky je ukulele sopránové. Sopránové ukulele je zároveň nejmenší a jeho běžné ladění je \textit{g,c,e,a} nebo \textit{a,d,f\textsuperscript{\#},h}.

Další ukulele jsou koncertní a tenorové, které jsou větší, ale ladí se stejně jako ukulele sopránové. Rozdíl je tedy v barvě zvuku a velikosti, tedy pohodlnosti držení.

Největší ukulele je barynotové, které je běžně laděné stejně jako vrchní čtyři struny kytary, tedy \textit{d,g,h,e}, a je vhodné pro hráče kteří již umí hrát na kytaru, právě kvůli stejnému ladění.

Další, méně obvyklé verze ukulele jsou šestistrunné, osmistrunné, desetistrunné a uke-benžo (benžolele), které vzniklo spojením ukulele a benža.


\section{Existující aplikace}
\label{sc:existing_apps}

\section{Dostupné technologie}
\label{sc:available_technologies}
Aplikace by měla být co nejdostupnější pro běžného uživatele, ideálně by tecy neměla vyžadovat stahování nebo instalaci. Z podstaty této aplikace to ani není potřeba. Jediná výhoda kterou přináší desktopová aplikace je přímý přístup k hardware a lepší výkon, jelikož ale tento typ aplikace nepotřebuje vykreslovat pokročilou 3D grafiku nebo spouštět složité algoritmy, tak je zbytečné se tím zabývat.

Nabízí se tedy webová aplikace nebo mobilní aplikace. Oba typy aplikací se dělí na dvě části, část kterou vidí a se kterou intereaguje uživatel nebo správce, která se nazývá frontend a část která obsluhuje práci s databází, zpracovává požadavky na změny, autorizuje uživatele, atd. Jeden backend přitom může obsluhovat frontend jak podobě webové, tak i mobilní aplikace. Rozdělíme si tedy sekce na \nameref{ss:backend}, \nameref{ss:web} a \nameref{ss:mobile}.

\subsection{Backend}
\label{ss:backend}
Backend tvoří páteř aplikací, zprostředkovává přístup k informacím, umožňuje úpravy a autorizuje uživatele. Z toho důvodu, je třeba dbát na kvalitu a hlavně bezpečnost kódu. Špatně autorizovaný uživatel, nebo nezabezpečená část databáze je velká bezpečnostní díra a tomu se musí předejít. Nedá se spoléhat na to, že uživatele ověřil frontend, jelikož se s daty na frontendu dá manipulovat (jak na webu, tak i v mobilu), a proto veškeré ověřování probíhá na serveru. Ověřování na frontendu tedy slouží jako čistě estetická záležitost.

Dalším důležitým faktorem je rychlost. Rychlost může ovlivnit několik faktorů, jako je třeba využití složitých algoritmů bez paralelizace, pomalý databázový server, nebo neoptimalizované databázové požadavky.

Při výběru je třeba tedy dbát hlavně na bezpečnost a rychlost. Avšak je třeba zvážit i tzv. code reuse, tedy částečné přepoužití kódu, které muže zjednodušit a zpřehlednit kód aplikace.

\subsubsection{Javascript}
Javascript je \textit{otevřený multiplatformní skriptovací jazyk pro tvorbu a přizpůsobení aplikací v podnikových sítích a na internetu.} \cite{netscapecommunicationscorporation_1995_press}, jako první implementovaný prohlížečem Netscape Navigator 2.0 a rychle adaptován ostatními prohlížeči.

Výhodou je multiplatformnost a jednoduchost vývoje a nasazení. Hlavní nevýhodou Javascriptu je, že Javascript je dynamicky typovaný, což vede k horší udržitelnosti kódu a odhalení většiny chyb až při běhu aplikace (absence statické kontroly). Tyto neduhy se dají odstranit použitím nějakou syntaktickou nadstavbou, jako je třeba Typescript nebo Flow, které přidávají statické typování proměnných a statickou kontrolu při překladu do Javascriptu.

Javascript se původně vyskytoval jen v prohlížeči, což znamená že nebyl použitelný pro vývoj desktopových a serverových aplikací. To se změnilo s příchodem Node.js. Node.js je prostředí, které umožňuje spouštět javascriptové skripty mimo prohlížeč.

\subsubsection{Java}
Staticky typovaný, multiplatformní programovací jazyk vycházející z jazyka C, zaštítěný firmou Oracle, Java, patří mezi nejpopulárnější a nejužívanější programovací jazyky \cite{stackexchangeinc_2019_stack} . Jeho hlavní užití je právě na straně serveru, a to v kombinaci s frameworkem Spring Boot \cite{jetbrainssro_2019_demographics} .

\subsubsection{C\# }
Jazyk velmi podobný Javě, vyvýjený firmou Microsoft. Jeho hlavní nevýhodou byla vysoká závislost na frameworku .NET, který až do příchodu alternativy .NET Core, byl spustitelný pouze na operačním systému Microsoft Windows.

\subsection{Web}
\label{ss:web}
Webová aplikace je pro uživatele nejpřístupnější formou, nemusí nic instalovat, stahovat, pouze otevře prohlížeč s webovou adresou a aplikaci má přístupnou a to jak na počítači, tak i na mobilu. Nevýhoda oproti mobilní aplikace je, že vyžaduje přístup k internetu (ne vždy, viz PWA **TODO**).

\subsubsection{Javascript}
Pro vývoj webové aplikace lze v Javascriptu zvolit z mnoha možností, ať už co se týče syntaktické nadstavby (Typescript, Flow), nebo knihoven a frameworků. Jelikož jich je velké množství, tak si jen stručně probereme tři nejpopulárnější a to React.JS, Angular a Vue.

React.JS je knihovna vyvýjená společností Facebook, která ho sama využívá pro tvorbu jejich aplikací jako je třeba Instagram, nebo WhatsApp. Jeho hlavní výhodou je možnost si spoustu věcí přizpůsobit k vlastní potřebě (např. routing, globalní uložiště dat, ...), což mimo jiné vede k menší velikosti výsledné aplikace a vyšší rychlosti. React sám o sobě není ani závislý na prohlížeči a může být využit např. pro vývoj aplikace spouštěné z příkazové řádky.

Angular je framework od společnosti Google, taktéž využívaný v řadě aplikací. Hlavní výhodou je, že Angular nabízí téměř vše co může programátor potřebovat, díky čemuž nemusí programátor řešit občasné potíže s kompatibilitou knihoven jako u Reactu, ale na druhou stranu je výsledná aplikace velká a pomalejší, jelikož obsahuje i nevyužívané funkce.

Vue je knihovna velmi podobná Reactu, sdílí spolu velkou část přístupů, ale Vue se zaměřuje na projekty malé až střední a React in na ty velké. Výhoda tedy je rychlost a jednoduchost, nevýhoda je, že ve velkém projektu se programátor, nebo tým programátorů může rychle začít ztrácet a zpomalí se vývoj.

\subsubsection{C\# }
V C\# se dá vyvýjet i frontend a to za pomoci frameworku ASP.NET, který závisí na frameworku .NET. V dnešní době se na nové projekty již příliž nevyužívá. Lze jej využít v kombinaci s Javascriptem a jeho knihovnami. Největší výhoda tohoto přístupu je striktní dodržení přístupu Model-View-Controler, nevýhodou jsou vyšší nároky na znalost obou technologií, což pro juniorního programátora může být problém.

\subsection{Mobil}
\label{ss:mobile}

\section{Cílová skupina}
\label{sc:target_audience}
Do cílové skupiny této práce se řadí kdokoliv kdo má zájem se naučit hrát na ukulele, celosvětově. To znamená, že uživatelem může být muž či žena od šesti do šedesáti let. Z toho se odvíjí požadavky na jazyk aplikace a uživatelskou přívětivost. Jelikož spektrum uživatelů je velmi rozsáhlé, tak uživatelské prostředí musí být jednoduché a přehledné. Webová aplikace umožní uživateli si aplikaci zobrazit jak na počítači, tak na mobilu či tabletu a zároveň nemá potřebu nic instalovat. Výchozím jazykem bude angličtina, protože je celosvětově rozšířenejší než čeština, ale aplikace musí být do budoucna připřavena mít více jazyků.



\begin{conclusion}
    Cílem práce bylo vytvořit aplikaci pro podporu výuky hry na ukulele v~souladu s~metodami softwarového inženýrství. Ve zkratce jsem uvedl, co je potřeba znát pro hraní na ukulele. Prozkoumal jsem již existující aplikace a zvážil jejich výhody a nevýhody. Dále jsem uvedl dostupné technologie a cílovou skupinu, pro kterou je aplikace tvořena. Na základě těchto poznatků jsem stanovil funkční a nefunkční požadavky, ze kterých jsem vytvořil případy užití, wireframy a diagram tříd. Následně jsem zvolil technologie, které jsem využil k~vytvoření aplikace. Popsal jsem metodiky použité k~návrhu architektury aplikace a určil logické členění částí aplikace. Na základě návrhu jsem implementoval prototyp aplikace a automatické testy aplikace. Poté jsem vytvořil testovací a nasazovací skripty, které pracují zcela automaticky.

    Prototyp aplikace je plně funkční, umožňuje vyhledávat akordy, strumming patterny, písně i autory. Uživatel má možnost se přihlásit a tím zpřístupnit nastavování písní jako oblíbené. Moderátor může přidávat, či upravovat písně a autory a administrátor může měnit role vytvořených účtů. Aplikace je dostupná v~anglickém jazyce.

    Aplikaci lze rozšířit o~další funkcionality, třeba vybrnkávání, či nastavování vlastního ladění ukulele. Vhodné by bylo aplikaci rozšířit o~další jazyky či vylepšit grafické pojetí. Dobrým nápadem je z~responzivní aplikace udělat \hyperref[sss:pwa]{progresivní webovou aplikaci}.

    Hlavní přínos práce spočívá v~přehlednosti a jednoduchosti aplikace, z~čehož benefitují hlavně uživatelé méně zkušení s~počítačem. Zvolené technologie umožnily rychle a efektivně vytvořit aplikaci, kterou je možné jednoduše rozšířit, či upravit. Jediná nevýhoda zvolených technologií je špatná kompatibilita, což znemožnilo integrační testování. Tento problém by však měl v~budoucnu s~rostoucí komunitou vymizet a aplikaci tedy bude možné o~dané testy rozšířit a tím zajistit konzistenci kvality.
\end{conclusion}

\bibliographystyle{csn690}
\bibliography{bybliography}

\appendix

\chapter{Seznam použitých zkratek}
% \printglossaries
\begin{description}
	\item[GUI] Graphical user interface
	\item[XML] Extensible markup language
\end{description}

\chapter{Obsah přiloženého CD}

%upravte podle skutecnosti

\begin{figure}
	\dirtree{%
		.1 readme.txt\DTcomment{stručný popis obsahu CD}.
		.1 build\DTcomment{adresář se spustitelnou formou implementace}.
		.2 react\DTcomment{adresář se spustitelným frontend serverem}.
		.2 nodejs\DTcomment{adresář se spustitelným backend serverem}.
		.1 src.
		.2 impl\DTcomment{zdrojové kódy implementace}.
		.2 thesis\DTcomment{zdrojová forma práce ve formátu \LaTeX{}}.
		.2 thesis.pdf\DTcomment{text práce ve formátu PDF}.
	}
\end{figure}

\end{document}
