% arara: xelatex
% arara: xelatex
% arara: xelatex

\documentclass[thesis=B,czech]{template/FITthesis}[2019/03/06]
\selectlanguage{czech}

\usepackage[utf8]{inputenc}

\usepackage{dirtree}

% % list of acronyms
\usepackage[acronym,nonumberlist,toc,numberedsection=autolabel]{glossaries}
\iflanguage{czech}{\renewcommand*{\acronymname}{Seznam pou{\v z}it{\' y}ch zkratek}}{}
\makeglossaries

\newcommand{\tg}{\mathop{\mathrm{tg}}} %cesky tangens
\newcommand{\cotg}{\mathop{\mathrm{cotg}}} %cesky cotangens

\department{Katedra softwarového inženýrství}
\title{Specializovaná webová aplikace pro výuku hry na ukulele}
\authorGN{Dan} %(křestní) jméno (jména) autora
\authorFN{Balarin} %příjmení autora
\authorWithDegrees{Dan Balarin} %jméno autora včetně současných akademických titulů
\author{Dan Balarin} %jméno autora bez akademických titulů
\supervisor{Ing. Marek Suchánek}
\acknowledgements{TODO\: Doplňte, máte-li komu a za co děkovat. V~opačném případě úplně odstraňte tento příkaz.}
\abstractCS{TODO\: V~několika větách shrňte obsah a přínos této práce v~češtině. Po přečtení abstraktu by se čtenář měl mít čtenář dost informací pro rozhodnutí, zda chce Vaši práci číst.}
\abstractEN{TODO\:Sem doplňte ekvivalent abstraktu Vaší práce v~angličtině.}
\placeForDeclarationOfAuthenticity{V~Praze}
\declarationOfAuthenticityOption{4} %volba Prohlášení (číslo 1-6)
\keywordsCS{TODO\: Nahraďte seznamem klíčových slov v češtině oddělených čárkou.}
\keywordsEN{TODO\: Nahraďte seznamem klíčových slov v angličtině oddělených čárkou.}
\website{https://github.com/danbalarin/bachelors-thesis} %volitelná URL práce, objeví se v tiráži - úplně odstraňte, nemáte-li URL práce

\begin{document}

\begin{introduction}
    \label{ch:introduction}
    \section{Motivace}
    V posledních letech roste popularita hry na hudební nástroje a s rozvojem informačních technologií s tím souvisí i snaha uživatelů učit se sami za pomoci veřejně dostupných aplikací. Pro začátečníky se doporučují nástroje buď drnkací (ukulele, kytara) nebo dechové (harmonika, flétna)~\cite{s_2016_the}~\cite{richardson_2019_top}.

    Ukulele oproti kytaře má tu výhodu, že je menší, má o dvě struny méně, struny tolik neřežou do prstů a je tedy jednodušší pro začátečníka. Oproti nástrojům dechovým má hlavně tu výhodu, že se na něj dá zahrát více populárních písní.

    Aktuálně existuje již několik aplikací, ať už webových nebo mobilních, které umožňují uživateli učit se hrát na ukulele sám doma. Problémy těchto aplikací jsou, že jsou buď placené, neobsahují všechny funkcionality, nebo nejsou uživatelsky přívětivé. Rozborem již existujicích řešení se podrobněji zabývá kapitola \nameref{ch:analysis}.

    K této problematice jsem se dostal, jelikož jsem se sám začal učit hrát na ukulele. Již existujicí aplikace mi nevyhovují a to převážně kvůli chybějícím funkcím, nebo neintuitivnímu ovládání. Z tohoto důvodu mi přišlo ideální vytvořit vlastní aplikaci, která bude obsahovat veškeré funkce, chybějící u již existujících aplikací, bude snadno ovladatelná a bude fungovat na počítači i chytrém telefonu.


    \section{Cíle práce}
    Hlavním cílem práce je navrhout, vytvořit a otestovat aplikaci, která umožní začátečníkovi naučit se hrát na ukulele, čehož lze dosáhnout několika způsoby, kterými se budeme zabývat v kapitole \nameref{ch:analysis}. Dílčím cílem je umožnit již zkušenému hráči vyhledat jednotlivé akordy nebo akordy a text k písni.

    Důležitou součástí aplikace je jednoduchost použití, uživatelská přívětivost a zároveň možnost použití některých složitějších mechanismů, jako je například výuka přechytávání akordů s možností zapnutí metronomu, nebo vyhledávání v několika různých množinách (např. akordy, písně a autoři).

    \section{Členění práce}
    Práce je členěna do kapitol odpovídajících krokům vývoje software podle principů metodologie unifikovaného vývoje~\cite[s.~51‑68]{arlow_2007_uml}.

    Kapitola \nameref{ch:analysis} se zabývá průzkumem již existujících aplikací, zjištění cílové skupiny, dostupných technologií a platforem a možnostmi, jak k danému problému přistoupit.

    \hyperref[ch:design]{Druhá kapitola práce}, se věnuje návrhu aplikace. Návrh aplikace je nedílnou součástí vývoje složitějších systémů a aplikací. Umožní programátorovi, nebo týmu programátorů, mít lepší představu o celém projektu, zjistit nedostatky ještě před implementací, nebo například stanovit místa v aplikaci kde může v budoucnu dojít k rozšíření a aplikaci na to řádně připravit. Tato kapitola obsahuje formální stanovení požadavků na aplikaci a vytvoření náhledů uživatelského rozhraní.

    Kapitola \nameref{ch:implementation} rozebírá prostředí v jakém probíhal vývoj aplikace, konečný výběr technologií, určení logického členění a architektury aplikace a některé aspekty zdrojového kódu samotné aplikace. Správně zvolená architektura aplikace usnadní vývoj teď i případné rozšiřování, a může umožnit přepoužití části aplikace při práci na aplikaci jiné.

    \hyperref[ch:testing]{Čtvrtá kapitola} se věnuje testovaním aplikace. Testování je naprosto nezbytné pro vývoj kvalitní aplikace. Vývojář díky tomu zjistí, jakými nedostatky aplikace trpí ještě před tím, než na ně narazí uživatel. Jelikož tyto nedostatky mohou být kritické, může díky nim docházet k úniku citlivých informací a tím poškodit uživatele. Proto se každý systém a aplikace testuje a to na několika úrovních.

    \hyperref[ch:ci_cd]{Poslední kapitola} probírá automatické testování a nasazování. Automatizace testů a nasazování zrychluje vývoj a celkovou workflow týmu. Tato automatizace deleguje spouštění testů a buildů aplikace na dedikovaný server, takže se vývojář může zatím věnovat něčemu jinému.
\end{introduction}

\begin{conclusion}
    Cílem práce bylo vytvořit aplikaci pro podporu výuky hry na ukulele v~souladu s~metodami softwarového inženýrství. Ve zkratce jsem uvedl, co je potřeba znát pro hraní na ukulele. Prozkoumal jsem již existující aplikace a zvážil jejich výhody a nevýhody. Dále jsem uvedl dostupné technologie a cílovou skupinu, pro kterou je aplikace tvořena. Na základě těchto poznatků jsem stanovil funkční a nefunkční požadavky, ze kterých jsem vytvořil případy užití, wireframy a diagram tříd. Následně jsem zvolil technologie, které jsem využil k~vytvoření aplikace. Popsal jsem metodiky použité k~návrhu architektury aplikace a určil logické členění částí aplikace. Na základě návrhu jsem implementoval prototyp aplikace a automatické testy aplikace. Poté jsem vytvořil testovací a nasazovací skripty, které pracují zcela automaticky.

    Prototyp aplikace je plně funkční, umožňuje vyhledávat akordy, strumming patterny, písně i autory. Uživatel má možnost se přihlásit a tím zpřístupnit nastavování písní jako oblíbené. Moderátor může přidávat, či upravovat písně a autory a administrátor může měnit role vytvořených účtů. Aplikace je dostupná v~anglickém jazyce.

    Aplikaci lze rozšířit o~další funkcionality, třeba vybrnkávání, či nastavování vlastního ladění ukulele. Vhodné by bylo aplikaci rozšířit o~další jazyky či vylepšit grafické pojetí. Dobrým nápadem je z~responzivní aplikace udělat \hyperref[sss:pwa]{progresivní webovou aplikaci}.

    Hlavní přínos práce spočívá v~přehlednosti a jednoduchosti aplikace, z~čehož benefitují hlavně uživatelé méně zkušení s~počítačem. Zvolené technologie umožnily rychle a efektivně vytvořit aplikaci, kterou je možné jednoduše rozšířit, či upravit. Jediná nevýhoda zvolených technologií je špatná kompatibilita, což znemožnilo integrační testování. Tento problém by však měl v~budoucnu s~rostoucí komunitou vymizet a aplikaci tedy bude možné o~dané testy rozšířit a tím zajistit konzistenci kvality.
\end{conclusion}

\bibliographystyle{csn690}
\bibliography{bybliography}

\appendix

\chapter{Seznam použitých zkratek}
% \printglossaries
\begin{description}
	\item[GUI] Graphical user interface
	\item[XML] Extensible markup language
\end{description}

\chapter{Obsah přiloženého CD}

%upravte podle skutecnosti

\begin{figure}
	\dirtree{%
		.1 readme.txt\DTcomment{stručný popis obsahu CD}.
		.1 build\DTcomment{adresář se spustitelnou formou implementace}.
		.2 react\DTcomment{adresář se spustitelným frontend serverem}.
		.2 nodejs\DTcomment{adresář se spustitelným backend serverem}.
		.1 src.
		.2 impl\DTcomment{zdrojové kódy implementace}.
		.2 thesis\DTcomment{zdrojová forma práce ve formátu \LaTeX{}}.
		.2 thesis.pdf\DTcomment{text práce ve formátu PDF}.
	}
\end{figure}

\end{document}
