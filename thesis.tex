% arara: xelatex: { shell: yes }
% arara: biber
% arara: xelatex: { shell: yes }
% arara: xelatex: { shell: yes }

\documentclass[thesis=B,czech]{template/FITthesis}[2019/03/06]
\selectlanguage{czech}

\usepackage[utf8]{inputenc}
\usepackage[final]{pdfpages}
\usepackage{dirtree}
\usepackage{minted, xcolor}
\usepackage{pifont}
\usemintedstyle{friendly}
\usepackage{csquotes}
\usepackage{xevlna}
\usepackage{pgfplots}

\usepackage[style=iso-numeric]{biblatex}
\addbibresource{bybliography.bib} 


% % list of acronyms
\usepackage[acronym,nonumberlist,toc,numberedsection=autolabel]{glossaries}
\iflanguage{czech}{\renewcommand*{\acronymname}{Seznam pou{\v z}it{\' y}ch zkratek}}{}
\makeglossaries

\newcommand{\tg}{\mathop{\mathrm{tg}}} %cesky tangens
\newcommand{\cotg}{\mathop{\mathrm{cotg}}} %cesky cotangens

\department{Katedra softwarového inženýrství}
\title{Specializovaná webová aplikace pro výuku hry na ukulele}
\authorGN{Dan} %(křestní) jméno (jména) autora
\authorFN{Balarin} %příjmení autora
\authorWithDegrees{Dan Balarin} %jméno autora včetně současných akademických titulů
\author{Dan Balarin} %jméno autora bez akademických titulů
\supervisor{Ing. Marek Suchánek}
\acknowledgements{Děkuji vedoucímu práce, Ing. Marku Suchánkovi, za odborné vedení a ochotu při tvorbě práce. Dále chci poděkovat přítelkyni a rodičům za podporu a schovívavost v~průběhu studia.}
\abstractCS{Bakalářská práce se zabývá vývojem webové aplikace podporující výuku hraní na ukulele. Aplikace je vyvíjena v~souladu s~metodami softwarového inženýrství. Zkoumá existující řešení, dostupné technologie a cílovou skupinu. Následně stanovuje formální požadavky na aplikaci a dané požadavky implementuje a testuje. V~poslední části rozebírá automatické testování a nasazování. Výsledkem je funkční aplikace umožňující výuku hraní na ukulele, která je díky zvolené architektuře snadno rozšiřitelná.}
\abstractEN{Bachelor thesis describes the development of web application for learning how to play on the ukulele. The application is developed according to methods of software engineering. Thesis researches existing solutions, available technologies, and target audience. After that, it sets formal requirements on application and implements and tests these requirements. The last part is focused on automatic testing and deployment. The result is a functional application for learning how to play on the ukulele, which is thanks to the good system architecture easily expandable.}
\placeForDeclarationOfAuthenticity{V~Praze}
\declarationOfAuthenticityOption{4} %volba Prohlášení (číslo 1-6)
\keywordsCS{ukulele, metronom, akordy, interaktivní výuka, webová aplikace, React, GraphQL, kontinuální integrace, automatické nasazování}
\keywordsEN{ukulele, metronome, chords, interactive lessons, web application, React, GraphQL, continuous integration, automatic deployment}
\website{https://github.com/danbalarin/bachelors-thesis} %volitelná URL práce, objeví se v tiráži - úplně odstraňte, nemáte-li URL práce


\newacronym{npm}{npm}{Node Package Manager}
\newacronym{pnp}{PnP}{Plug'n Play}
\newacronym{nra}{NRA}{Node Resolution Algorithm}
\newacronym{api}{API}{Application Programming Interface}
\newacronym{seo}{SEO}{Search Engine Optimalization}
\newacronym{rest}{REST}{Representational state transfer}
\newacronym{ssr}{SSR}{Server Side Rendering}
\newacronym{orm}{ORM}{Object Reloation Mapping}
\newacronym{fcp}{FCP}{First Contentfull Paint}


\newcommand{\smallindentsize}{1cm}

\newenvironment{usecase}[2]{
    \noindent
    \paragraph{#1}
    \label{#2}
    \begin{smallindent}{}
        }{
    \end{smallindent}
}

\newenvironment{scenario}[2]{
    \paragraph{#1}
    \label{#2}
    \begin{enumerate}
        }{
    \end{enumerate}
}

\newenvironment{requirment}[2]{
    \noindent \begin{minipage}{\textwidth}
        \paragraph{#1}
        \label{#2}
        \begin{smallindent}{}
            }{
        \end{smallindent}
    \end{minipage}
}

\newenvironment{smallindent}{%
    \begin{adjustwidth}{\smallindentsize}%
        }{%
    \end{adjustwidth}%
    \bigskip
}


\graphicspath{{./assets/use_case_mode.png}}

\begin{document}

\begin{introduction}
    TODO Introduction
	%sem napište úvod Vaší práce
\end{introduction}

\chapter{Zavedení základních pojmů}
\label{ch:glosary}
Dříve, než se práce začne zabývat problematikou více do hloubky, je třeba si stanovit některé základní pojmy z teorie hudby a stručný popis ukulele. Po více informací bych doporučil odbornou literaturu, např. \cite[Hudební slovník pro každého]{jirivyslouzil_1995_hudebni}.

\section{Základ teorie hudby}
Základ hudby je tón, zvuk s periodickým kmitáním o určité frekvenci, příkladem může být tón \emph{C}. Následně hudební stupnice určuje počet tónů a půltónů a jejich frekvenční vzdálenost. Práce obsahuje pouze stupnice durové a mollové a to v souvislosti s akordy, příkladem může být stupnice \emph{C dur}. Akord je souzvuk několika tónů, u ukulele to je souzvuk čtyř tónů, jeden pro každou strunu. Označení akordů vychází z tónů a vzdálenostmi mezi nimi, které se většinou řídí nějakou stupnicí, např. akord \emph{Emi}, neboli \emph{e moll} se řídí mollovou stupnicí.

\section{Popis ukulele}
\label{sc:ukulele_description}
Ukulele je čtyřstrunný drnkací hudební nástroj, vzhledem podobný malé kytaře. Ukulele původně pochází z~Havajských ostrovů kde vzniklo z~nástroje zvaného Braguinha a o~původu jeho jména se vedou spekulace~\cite{rek_2008_kola}.

Skládá se z dvou hlavních částí - těla a krku. Krk se následně skládá z hmatníku, který tvoří pražce a hlavice s ladící mechanikou. Struny jsou nataženy od kobylky až k ladící mechanice a zpravidla mají různé ladění. Běžné ladění udává ladění jednotlivých strun od shora dolů.

\subsection{Hra na ukulele}
Samotné hraní je velmi podobné hře na kytaru –⁠ ukazováček, prostředníček, prsteníček a malíček levé ruky přitiskávají struny mezi pražci, čímž určují akord a prsty pravé ruky přejíždějí po strunách někde na pomezí krku a těla ukulele. Způsobů, jak držet akordy, je několik a většinou záleží na dalším, resp. předchozím akordu, aby si hráč zjednodušil přechod. Stejně tak je i více způsobů jak přejíždět prsty po strunách. Prsty mohou jet dolů, nebo nahoru a to břískem nebo nehtem prstu, či lehce klepnout do těla a tím zároveň utlumit struny. Různým kombinacím těchto pohybů se říká strumming pattern.

\subsection{Typy ukulele}
Ukulele má několik variant, které se odlišují velikostí nástroje a barvou zvuku. Nejběžnější a zároveň doporučované pro začátečníky je ukulele sopránové. Sopránové ukulele je zároveň nejmenší a jeho běžné ladění je \textit{g,c,e,a} nebo \textit{a,d,f\textsuperscript{\#},h}.

Další ukulele jsou koncertní a tenorové, které jsou větší, ale ladí se stejně jako ukulele sopránové. Rozdíl je tedy v~barvě zvuku a velikosti, tedy pohodlnosti držení.

Největší ukulele je barytonové, které je běžně laděné stejně jako vrchní čtyři struny kytary, tedy \textit{d,g,h,e}, a je vhodné pro hráče kteří již umí hrát na kytaru, právě kvůli stejnému ladění.

Další, méně obvyklé, verze ukulele jsou šestistrunné, osmistrunné, desetistrunné a uke-bendžo (bendžolele), které vzniklo spojením ukulele a benža.

\begin{figure}
    \centering
    \includegraphics[width=\textwidth]{assets/ukulele.png}
    \caption[Příklad koncertního ukulele]{Příklad koncertního ukulele \cite{cordobaguitars_2020_ukulele}}
    \label{fig:class_diagram}
\end{figure}

\chapter{Analýza}
\label{ch:analysis}
Proces vývoje aplikace začíná u analýzy. V rámci analýzy se zkoumá daná problematika, kterou má aplikace řešit, již existující řešení, cílová skupina a možnosti řešení. Tato část je důležitá hlavně z toho důvodu, abychom náhodou neřešili problém, který už někdo vyřešil za nás, nebo nevytvářeli aplikaci pro jinou cílovou skupinu, než bychom měli.

\section{Hra na ukulele}
\label{sc:ukulele_playing}

\section{Existující aplikace}
\label{sc:existing_apps}
Mezi alternativy k této práci lze řadit aplikace zabívající se výukou hry na ukulele nebo kytaru, zpěvníky a metronomy.

\subsection{Výuka hry na ukulele}
\label{ss:existing_teaching_apps}
Většina materiálů pro výuku hry na ukulele je ve formě knih, nebo kurzů.
Kurzy jsou buď osobní nebo online. Mezi největší a nejznámější online aplikace poskytující různé kurzy patří Udemy, který má i kurz hry na ukulele pro začátečníky \cite{puchmayr_complete}. Alternativou ke knihám a kurzům jsou ještě články nebo krátká nenavazující videa.

Články lze najít například na webu \href{www.kytary.cz}{\emph{kytary.cz}} nebo \href{www.ukuguides.com}{\emph{ukuguides.com}}, kde to je zpracované formou tipů čemu věnovat pozornost a čemu se vyvarovat. Krátká videa zabívající se tipy pro hraní, nebo samotnými skladbami lze najít převážně na portálu \href{www.youtube.com}{\emph{youtube.com}}. Mezi tvůrce s nejvíce shlednutými videi měsíčně patří např. Elise Ecklund nebo John Atkins vystupující pod přezdívkou The Ukulele Teacher. Oba autoři mají videa na veškerá témata hry na ukulele, od jeho výběru až po samotné písně s akordy a jak je hrát. John Atkins je navíc jedním z tvůrců aplikace The Ukulele App, která slouží hlavně pro výuku hry, ale neobsahuje žádné písně s akordy a vetšina funkcionalit je zpoplatněna.

\subsection{Zpěvníky}
\label{ss:songbooks}
Zpěvníky jsou ve většině případů tištěné knihy, pdf soubory k tisku nebo v podobě webové nebo mobilní aplikace. Knižních zpěvníků ukulele je méně než kytarových a ukulele zpěvníků s českými písněmi je minimum. Soubory k tisku se dají sehnat i s českými písněmi, což částečně kompenzuje jejich absenci v podobě knižní.

Mezi významné webové aplikace se řadí \href{www.ukulele-tabs.com}{\emph{ukulele-tabs.com}} nebo \href{www.ukutabs.com}{\emph{ukutabs.com}}. Mají možnost řazení a vyhledávání ve velkém repertoáru písní. Mezi jejich úskalí patří však absence responzivity (Ukutabs) a absence metronomu, či případné doporučení strumming patternu.

Nejstahovanější aplikace na mobilní telefony jsou např. Ukulele Tabs \& Chords, nebo Ukulele Chords Pocket pro Android a The Ukulele App pro Android a iOS.

\subsection{Metronomy}
\label{ss:metronomes}
Metronomy jsou buď ve formě fyzického zařízení, nebo aplikace, ať už webové nebo mobilní.

Základní metronom poskytuje i webový vyhledávač \href{www.google.com}{\emph{google.com}}, který ovšem umožňuje pouze nastavení tempa. Mezi jeden z nejlépších patří metronom webu \href{www.musicca.com}{\emph{musicca.com}}, který umožňuje mimo nastavení tempa i nastavení dob a možnost hrát každý druhý takt, a metronom na webu \href{www.douglasniedt.com}{\emph{douglasniedt.com}}, který sice umožňuje pouze 4/4 takt, ale zase zobrazuje doby v podobě větších dlaždic, které je mozné vnímat i periferním viděním, takže uživatel nemusí vnímat tempo jen pomocí zvuků.

Mobilní aplikace zastupujeMetronome Beats a Tuner \& Metronome pro Android a The Metronome by Soundbrenner pro iOS a Android.


\section{Dostupné technologie}
\label{sc:available_technologies}
Aplikace by měla být co nejdostupnější pro běžného uživatele, ideálně by tecy neměla vyžadovat stahování nebo instalaci. Z podstaty této aplikace to ani není potřeba. Jediná výhoda kterou přináší desktopová aplikace je přímý přístup k hardware a lepší výkon, jelikož ale tento typ aplikace nepotřebuje vykreslovat pokročilou 3D grafiku nebo spouštět složité algoritmy, tak je zbytečné se tím zabývat.

Nabízí se tedy webová aplikace nebo mobilní aplikace. Oba typy aplikací se dělí na dvě části, část kterou vidí a se kterou intereaguje uživatel nebo správce, která se nazývá frontend a část která obsluhuje práci s databází, zpracovává požadavky na změny, autorizuje uživatele, atd. Jeden backend přitom může obsluhovat frontend jak podobě webové, tak i mobilní aplikace. Rozdělíme si tedy sekce na \nameref{ss:backend}, \nameref{ss:web} a \nameref{ss:mobile}.

\subsection{Backend}
\label{ss:backend}
Backend tvoří páteř aplikací, zprostředkovává přístup k informacím, umožňuje úpravy a autorizuje uživatele. Z toho důvodu, je třeba dbát na kvalitu a hlavně bezpečnost kódu. Špatně autorizovaný uživatel, nebo nezabezpečená část databáze je velká bezpečnostní díra a tomu se musí předejít. Nedá se spoléhat na to, že uživatele ověřil frontend, jelikož se s daty na frontendu dá manipulovat (jak na webu, tak i v mobilu), a proto veškeré ověřování probíhá na serveru. Ověřování na frontendu tedy slouží jako čistě estetická záležitost.

Dalším důležitým faktorem je rychlost. Rychlost může ovlivnit několik faktorů, jako je třeba využití složitých algoritmů bez paralelizace, pomalý databázový server, nebo neoptimalizované databázové požadavky.

Při výběru je třeba tedy dbát hlavně na bezpečnost a rychlost. Avšak je třeba zvážit i tzv. code reuse, tedy částečné přepoužití kódu, které muže zjednodušit a zpřehlednit kód aplikace.

\subsubsection{Javascript}
Javascript je \textit{otevřený multiplatformní skriptovací jazyk pro tvorbu a přizpůsobení aplikací v podnikových sítích a na internetu.} \cite{netscapecommunicationscorporation_1995_press}, jako první implementovaný prohlížečem Netscape Navigator 2.0 a rychle adaptován ostatními prohlížeči.

Výhodou je multiplatformnost a jednoduchost vývoje a nasazení. Hlavní nevýhodou Javascriptu je, že Javascript je dynamicky typovaný, což vede k horší udržitelnosti kódu a odhalení většiny chyb až při běhu aplikace (absence statické kontroly). Tyto neduhy se dají odstranit použitím nějakou syntaktickou nadstavbou, jako je třeba Typescript nebo Flow, které přidávají statické typování proměnných a statickou kontrolu při překladu do Javascriptu.

Javascript se původně vyskytoval jen v prohlížeči, což znamená že nebyl použitelný pro vývoj desktopových a serverových aplikací. To se změnilo s příchodem Node.js. Node.js je prostředí, které umožňuje spouštět javascriptové skripty mimo prohlížeč.

\subsubsection{Java}
Staticky typovaný, multiplatformní programovací jazyk vycházející z jazyka C, zaštítěný firmou Oracle, Java, patří mezi nejpopulárnější a nejužívanější programovací jazyky \cite{stackexchangeinc_2019_stack} . Jeho hlavní užití je právě na straně serveru, a to v kombinaci s frameworkem Spring Boot \cite{jetbrainssro_2019_demographics} .

\subsubsection{C\# }
Jazyk velmi podobný Javě, vyvýjený firmou Microsoft. Jeho hlavní nevýhodou byla vysoká závislost na frameworku .NET, který až do příchodu alternativy .NET Core, byl spustitelný pouze na operačním systému Microsoft Windows.

\subsection{Web}
\label{ss:web}
Webová aplikace je pro uživatele nejpřístupnější formou, nemusí nic instalovat, stahovat, pouze otevře prohlížeč s webovou adresou a aplikaci má přístupnou a to jak na počítači, tak i na mobilu. Nevýhoda oproti mobilní aplikace je, že vyžaduje přístup k internetu (ne vždy, viz \ref{sss:pwa}).

\subsubsection{Javascript}
Pro vývoj webové aplikace lze v Javascriptu zvolit z mnoha možností, ať už co se týče syntaktické nadstavby (Typescript, Flow), nebo knihoven a frameworků. Jelikož jich je velké množství, tak si jen stručně probereme tři nejpopulárnější a to React.JS, Angular a Vue.

React.JS je knihovna vyvýjená společností Facebook, která ho sama využívá pro tvorbu jejich aplikací jako je třeba Instagram, nebo WhatsApp. Jeho hlavní výhodou je možnost si spoustu věcí přizpůsobit k vlastní potřebě (např. routing, globalní uložiště dat, \ldots{}), což mimo jiné vede k menší velikosti výsledné aplikace a vyšší rychlosti. React sám o sobě není ani závislý na prohlížeči a může být využit např. pro vývoj aplikace spouštěné z příkazové řádky.

Angular je framework od společnosti Google, taktéž využívaný v řadě aplikací. Hlavní výhodou je, že Angular nabízí téměř vše co může programátor potřebovat, díky čemuž nemusí programátor řešit občasné potíže s kompatibilitou knihoven jako u Reactu, ale na druhou stranu je výsledná aplikace velká a pomalejší, jelikož obsahuje i nevyužívané funkce.

Vue je knihovna velmi podobná Reactu, sdílí spolu velkou část přístupů, ale Vue se zaměřuje na projekty malé až střední a React in na ty velké. Výhoda tedy je rychlost a jednoduchost, nevýhoda je, že ve velkém projektu se programátor, nebo tým programátorů může rychle začít ztrácet a zpomalí se vývoj.

\subsubsection{C\# }
V C\# se dá vyvýjet i frontend a to za pomoci frameworku ASP.NET, který závisí na frameworku .NET. V dnešní době se na nové projekty již příliž nevyužívá. Lze jej využít v kombinaci s Javascriptem a jeho knihovnami. Největší výhoda tohoto přístupu je striktní dodržení přístupu Model-View-Controler, nevýhodou jsou vyšší nároky na znalost obou technologií, což pro juniorního programátora může být problém.

\subsection{Mobil}
\label{ss:mobile}
Mobilním technologiím se budeme v této práci zabývat jen okrajově, protože cílem práce je webová aplikace. Přesto je jistá možnost jak z webové aplikace vytvořit aplikaci mobilní nebo z nějaké části sdílet kód mezi webovou a mobilní aplikací.

\subsubsection{Responzivní webová aplikace}
\label{sss:responsive_web_app}
První a nejjednodušší možností jak vytvořit mobilní aplikaci je pouze optimalizovat aplikaci webovou tak, aby se dala prohlížet i na mobilu. Hlavní výhodou je nízká náročnost vývoje takové aplikace (v porovnání s tvorbou celé mobilní aplikace), nevýhodou však je, že aplikace nemá přístup k některým částem zařízení, třeba stavu baterie, gyroskopu, poloze a dalším.

\subsubsection{Progressive Web App}
\label{sss:pwa}
Progressive Web App, nebo Progresivní Webová Aplikace, je způsob jak z mobilní aplikace udělat aplikaci mobilní bez nutnosti vytvářet přímo aplikaci webovou. Tento způsob je kombinací \ref{sss:responsive_web_app}, manifestu a service workeru. Manifest je soubor udávající informace pro mobilní prohlížeč, jak se aplikace má jmenovat, popis, ikony, barevné schéma, autor atd. Jedná se o obdobu manifestu přímo mobilní aplikace. Service worker je specialní skript, který běží na pozadí a umožňuje načítat data, přistupovat k notifikacím nebo poloze (na mobilu i webu) atd. V případě, že webová aplikace splňuje všechny zmíněné náležitosti, pak když uživatel vstoupí na web pomocí jednoho z prohlížečů, které podporují \hyperref[PWA]{sss:pwa}, pak má možnost si tu aplikaci přidat na plochu mobilu a spouštět stejně jako nainstalovanou aplikaci. Takto nainstalovaná aplikace funguje i bez přístupu k internetu, ale je to v podstatě stažený web spustěný v prohlížeči. Hlavní nevýhodou je absence možnosti práce přímo s hardwarem telefonu, takže to není vhodné např. pro mobilní hry. Dalším problémem je podpora na zařízeních s OS iOS, ačkoliv Apple je původním tvůrcem této myšlenky \cite{ritchie_2018_app}, tak aplikace na iOS nemají přístup k tolika informacím a možnostem jako ty na Androidu.

\subsubsection{Hybridní aplikace}
Hybridní aplikace jsou posledním mezikrokem mezi mobilní a webovou aplikací. Jedná se o aplikaci webovou, která je později zkompilovaná i s jádrem prohlížeče a může být přidaná na Android Play Store, resp. iOS App Store. Takto vytvořená aplikace má již přístup ke všem částem telefonu, stejně jako aplikace nativní, ale nevýhoda je velikost balíčku potažmo aplikace, která obsahuje v sobě i část prohlížeče a aplikace má vyšší nároky na RAM telefonu.

\subsubsection{Nativní aplikace}
Poslední kategorie jsou aplikace nativní, tedy přímo určené pouze na mobilní zařízení. Existuje mnoho programovacích jazyků, které lze při vývoji použít, ať už to je Java nebo Kotlin pro Android nebo Objective C a Swift pro iOS. Jejich nevýhoda je, že když programátor chce vytvořit mobilní aplikaci, tak musí vytvořit dvě oddělené aplikace, pro každou platformu zvlášť (V případě že chce podporovat pouze Android a iOS). Z toho důvodu exitují různé frameworky, zde si zmíníme pouze jeden a to React Native.

Jedná se o framework syntaxem velmi podobný frameworku ReactJS a důvod proč zmínit zrovna tento a ne ostatní, je možnost přepoužití kódu z webové aplikace. Jisté části aplikace napsané v ReactJS lze bez úpravy použít i v React Native a obráceně. Tedy aplikace psaná za použití React Native má stejné benefity jako aplikace nativní, ale zároveň v ní lze přepoužít části aplikace webové. Příklad takového přepoužití zde \cite{sepulveda_2017_share}.



\section{Cílová skupina}
\label{sc:target_audience}



\chapter{Návrh}
\label{ch:design}

\section{Funkční požadavky}
\label{sc:functional_req}

\begin{requirment}{FP01 Metronom}{FP01}
    Aplikace zobrazí uživateli metronom. Uživatel má možnost zapnout či vypnout metronom a zvolit si rychlost.
\end{requirment}

\begin{requirment}{FP02 Akordy}{FP02}
    Aplikace umožní uživateli si zobrazit libovolnou podmnožinu akordů.
\end{requirment}

\begin{requirment}{FP03 Strumming pattern}{FP03}
    Aplikace zobrazí uživateli strumming pattern a metronom. Uživatel může ovládat metronom dle \hyperref[FP01]{FP01} a zobrazení strumming patternu reaguje na nastavení rychlosti metronomu.
\end{requirment}

\begin{requirment}{FP04 Vyhledávání}{FP04}
    Aplikace umožní uživateli vyhledávat písně, akordy a strumming patterny.
\end{requirment}

\begin{requirment}{FP05 Přechytávání akordů}{FP05}
    Aplikace umožní uživateli si vytvořit vlastní seznam akordů s~volbou nastavení metronomu dle \hyperref[FP01]{FP01} a akordů dle \hyperref[FP02]{FP02}.
    Přihlášený uživatel má možnost si dané nastavení uložit pod libovolným jménem.
\end{requirment}

\begin{requirment}{FP06 Přechytávání akordů dle písně}{FP06}
    Aplikace zobrazí uživateli přechytávání akordů dle \hyperref[FP05]{FP05} s~přednastavenými parametry podle vybrané písně.
\end{requirment}

\begin{requirment}{FP07 Zobrazení písně}{FP07}
    Aplikace zobrazí uživateli text písně, akordy písně a strumming pattern dle \hyperref[FP03]{FP03}. Uživatel má možnost přechodu na přechytávání akordů dle \hyperref[FP06]{FP06} a případně odkaz na přehrání písně na serveru třetí strany (např. youtube, spotify).
\end{requirment}

\begin{requirment}{FP08 Registrace}{FP08}
    Aplikace umožní zaregistrovat nového uživatele.
\end{requirment}

\begin{requirment}{FP09 Přihlášení}{FP09}
    Aplikace umožní uživateli se přihlásit.
\end{requirment}

\begin{requirment}{FP22 Odhlášení}{FP22}
    Aplikace umožní odhlásit přihlášeného uživatele.
\end{requirment}

\begin{requirment}{FP23 Úprava profilu}{FP23}
    Aplikace umožní přihlášenému uživateli měnit svoje údaje.
\end{requirment}

% \noindent \begin{minipage}{\textwidth}
%     \paragraph{FP13 Nastavení ladění} \label{FP13}
%     \begin{smallindent}{}
%         Aplikace nastaví ladění ukulele přihlášeného uživatele podle vstupu od uživatele.
%     \end{smallindent}
% \end{minipage}

\begin{requirment}{FP24 Označení písně jako oblíbená}{FP24}
    Aplikace umožní příhlášenému uživateli označit píseň jako oblíbenou.
\end{requirment}

\begin{requirment}{FP41 Vytvoření písně}{FP41}
    Aplikace umožňuje moderátorovi vytvořit nový záznam o~písni.
\end{requirment}

\begin{requirment}{FP42 Úprava písně}{FP42}
    Aplikace umožňuje moderátorovi upravit existující záznam o~písni.
\end{requirment}

\begin{requirment}{FP61 Úprava rolí uživatele}{FP61}
    Aplikace umožňuje administrátorovi změnit role uživatele.
\end{requirment}


\section{Nefunkční požadavky}
\label{sc:non_func_req}

\noindent \begin{minipage}{\textwidth}
    \paragraph{NP01 Kompatibilita}
    \begin{smallindent}{}
        Aplikace je dostupná přes webové rozhraní. Webové rozhraní poskytuje podporu pro prohlížeče Mozilla Firefox od verze 68, Google Chrome od verze 79 a Safari od verze 12.
    \end{smallindent}
\end{minipage}


\noindent \begin{minipage}{\textwidth}
    \paragraph{NP02 Podpora menších obrazovek}
    \begin{smallindent}{}
        Aplikace je responzivní, tedy se adaptuje na velikost obrazovky uživatele a podporuje dané prohlížeče na mobilních telefonech.
    \end{smallindent}
\end{minipage}

\noindent \begin{minipage}{\textwidth}
    \paragraph{NP03 SEO}
    \begin{smallindent}{}
        Aplikace je SEO-friendly, tedy splňuje všechny náležitosti pro internetové vyhledávače jako jsou Google nebo Seznam. Tyto náležitosti jsou testovány pomocí nástroje Google Lighthouse \cite{googlellc_2019_lighthouse}, kde aplikace musí dosáhnout celkového bodového ohodnocení alespoň 90 ze 100.
    \end{smallindent}
\end{minipage}

\noindent \begin{minipage}{\textwidth}
    \paragraph{NP04 API}
    \begin{smallindent}{}
        Aplikace vystavuje soukromé API a to ve formátu GraphQL a REST.
    \end{smallindent}
\end{minipage}

\noindent \begin{minipage}{\textwidth}
    \paragraph{NP05 Rozšiřitelnost}
    \begin{smallindent}{}
        Aplikace je díky správnému návrhu a plánování možností rozšíření lehce rozšiřitelná.
    \end{smallindent}
\end{minipage}

\noindent \begin{minipage}{\textwidth}
    \paragraph{NP06 Lokalizace}
    \begin{smallindent}{}
        Aplikace je dostupná v českém jazyce.
    \end{smallindent}
\end{minipage}



\section{Případy užití}
\label{sc:use_cases}


\chapter{Implementace}
\label{ch:implementation}


\chapter{Testování}
\label{ch:testing}


\section{Budoucí vývoj}
\label{sc:upcomming_development}
Aplikace je nyní ve fázi funkčního prototypu, jsou implementovány všechny základní požadavky, ale stále je zde prostor na zlepšení. V následujících sekcích jsou rozebrány některé takové faktory.

\subsubsection{Data}
To co nejvíce aplikaci schází v tuto dobu jsou funkční data. Tím jsou myšleny písně s texty, akordy atd. Tyto data se dají jednoduše najít, ale pro osobní použití. Bohužel není žádné API poskytující seznam písní s textem a akordy pro ukulele. Tato data by se dala získat buď přepisem z jiných stránek či jejich scrapovaním, což by už mělo přesah do autorského zákona. Další možnost je kontaktovat přímo vydavatele a uzavřít spolupráci, která avšak nejspíš bude velmi finančně náročná. Poslední možnost by byla uzavřít spolupráci s již existujícímí aplikacemi a získat data od nich.

\subsubsection{Úpravy aplikace}
Hlavním neduhem stávající aplikace je hlavně design. Aplikace je sice ve dvou barevných schématech, ty však nejsou

\subsubsection{Nové funkce}
\subsubsection*{Překlad}

\subsubsection*{Podpora vybrnkávání}

\subsubsection*{Nastavení ladění}
Tato funkcionalita je témeř imlementována. Komponenta vykreslovaného akordu dostává na vstupu mimo akordu k vykreslení i aktuální ladění, které je vždy nastaveno na \emph{gcea}. Rozšíření by tedy spočívalo v přidání možnosti uživateli si toto ladění změnit, případně uložit pro příští návštěvy.



\chapter{Kontinuální integrace a nasazení}
\label{ch:ci_cd}
Kontinuální integrace, dodání a nasazení jsou přístupy k~automatizaci jednotlivých procesů, které jsou mezi vývojem a nasazením aplikace. V~rámci této práce si pouze nastíníme, co jednotlivé výrazy znamenají.

Kontinuální integrace se zabývá automatickou validací kódu pomocí automatizovaných testů. Tyto testy jsou spuštěny ve chvíli kdy programátor nahraje do systému správy verzí (\acrshort{scm}). Tyto automatické testy ověří funkčnost, správnost a další požadavky na aplikaci a následně zpětně informují vývojáře, resp. tým o~výsledcích testů. V~praxi se to běžně využívá před zapojením nové části kódu do celé aplikace. Výhodou tohoto přístupu je minimalizování chyb a zvýšení produktivity programátora. V~této práci kontinuální integrace byla realizována pomocí \emph{github actions}. Jelikož je celá aplikace open source a uložena na serveru \href{www.github.com}{github.com}, tak má nárok na využití systému github actions, který právě umožňuje automatické spouštění testů. Konkrétní testy prováděné automaticky jsou~\hyperref[sc:unit_tests]{jednotkové} a následně smoke build storybooku, který pouze ověří, zdali lze sestavit, tedy zdali aplikace lze sestavit. \cite[s.~7]{rossel_2017_continuous}

Kontinuální dodání a nasazení jsou pojmy velmi podobné. Kontinuální dodání vytvoří balíčky aplikace připravené k~nasazení a kontinuální nasazení je rovnou i nasadí. Jak dodání, tak nasazení je spouštěno automaticky po úspěšném provedení kontinuální integrace, nebo jiných krocích. V~praxi to může být manuální spuštění na konci sprintu nebo automaticky po zařazení kódu do hlavní větve. \cite[s.~18]{rossel_2017_continuous} Při tvorbě této aplikace bylo využito obojího. Při každé změně k~hlavní větvi jsou spuštěny dva nasazovací skripty. První zajišťuje automatické sestavení dokumentace a statické verze storybooku a nasazení na \emph{github pages}. Tato dokumentace je veřejně dostupná a hostovaná na serveru github\footnote{\href{https://danbalarin.github.io/ukulele-learning-site/}{danbalarin.github.io/ukulele-learning-site/}}. Druhý sestavuje výslednou aplikaci, vytváří docker kontejnery a následně spustí aktualizaci na vzdáleném serveru. Docker kontejnery jsou vytvořeny dva, jeden pro backend a druhý pro frontend a jsou uloženy na veřejném docker repozitáři\footnote{\href{https://hub.docker.com/repository/docker/kenny11/uls-react}{hub.docker.com/repository/docker/kenny11/uls-react}}\footnote{\href{https://hub.docker.com/repository/docker/kenny11/uls-nodejs}{hub.docker.com/repository/docker/kenny11/uls-nodejs}}. Následně se skript připojí přes ssh k~vzdálenému serveru, spustí jednoduchý skript, který zastaví běžící instance, smaže, nahraje nové a spustí je. Jako poslední krok je spuštěn jednoduchý smoke test~\ref{code:smoke}, který ověří, zdali frontend odpovídá na portu 80, a backend odpovídá na portu 4000.

\begin{figure}[h!]
    \centering
    \begin{minted}{bash}
curl -sSL --max-time 5 -D - $URL:$PORT -o /dev/null
    \end{minted}
    \caption{Smoke test ověřující funkčnost http serveru}
    \label{code:smoke}
\end{figure}


\begin{conclusion}
    Cílem práce bylo vytvořit aplikaci pro podporu výuky hry na ukulele v~souladu s~metodami softwarového inženýrství. Ve zkratce jsem uvedl, co je potřeba znát pro hraní na ukulele. Prozkoumal jsem již existující aplikace a zvážil jejich výhody a nevýhody. Dále jsem uvedl dostupné technologie a cílovou skupinu, pro kterou je aplikace tvořena. Na základě těchto poznatků jsem stanovil funkční a nefunkční požadavky, ze kterých jsem stanovil případy užití, wireframy a diagram tříd. Následně jsem zvolil technologie, které jsem využil k~vytvoření aplikace. Popsal jsem metodiky použité k~návrhu architektury aplikace a určil logické členění částí aplikace. Na základě návrhu jsem implementoval prototyp aplikace a automatické testy aplikace. Poté jsem vytvořil testovací a nasazovací skripty, které pracují zcela automaticky.

    Prototyp aplikace je plně funkční, umožňuje vyhledávat akordy, strumming patterny, písně i autory. Uživatel má možnost se přihlásit a tím zpřístupnit nastavování písní jako oblíbené. Moderátor může přidávat, či upravovat písně a autory a administrátor může měnit role vytvořených účtů. Aplikace je dostupná v~anglickém jazyce.

    Aplikaci lze rozšířit o~další funkcionality, třeba vybrnkávání, či nastavování vlastního ladění ukulele. Vhodné by bylo aplikaci rozšířit o~další jazyky či vylepšit grafické pojetí. Dobrým nápadem je z~responzivní aplikace udělat PWA.

    Hlavní přínos mé práce spočívá hlavně v~přehlednosti a jednoduchosti aplikace, z~čehož benefitují hlavně uživatelé méně zkušení s~počítačem. Zvolené technologie umožnily rychle a efektivně vytvořit aplikaci, kterou je možné jednoduše rozšířit, či upravit. Jediná nevýhoda zvolených technologií je špatná kompatibilita, což znemožnilo integrační testování. Tento problém by však měl v~budoucnu s~rostoucí komunitou vymizet a aplikaci tedy bude možné o~dané testy rozšířit a tím zajistit konzistenci kvality.
\end{conclusion}

\setcounter{biburllcpenalty}{9000}
\printbibliography

\appendix

\input{chapters/91_abbrevations.tex}

\chapter{Obsah přiloženého CD}

%upravte podle skutecnosti

\begin{figure}
	\dirtree{%
		.1 readme.txt\DTcomment{stručný popis obsahu CD}.
		.1 build\DTcomment{adresář se spustitelnou formou implementace}.
		.2 react\DTcomment{adresář se spustitelným frontend serverem}.
		.2 nodejs\DTcomment{adresář se spustitelným backend serverem}.
		.1 src.
		.2 impl\DTcomment{zdrojové kódy implementace}.
		.2 thesis\DTcomment{zdrojová forma práce ve formátu \LaTeX{}}.
		.2 thesis.pdf\DTcomment{text práce ve formátu PDF}.
	}
\end{figure}

\end{document}
